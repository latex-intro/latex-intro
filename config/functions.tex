
\newenvironment{fframe}{\begin{frame}[fragile,environment=specialframe]}{\end{frame}}
\newenvironment{codeblock}[1][]{\smallskip\begin{block}{Code: #1}}{\end{block}}

\definecolor{pkg}{HTML}{FF7F00}
\definecolor{umg}{HTML}{32CD32}

\newcommand{\R}{\mathbb{R}}

\newcommand{\upk}[1]{\textrm{\textbackslash usepackage\{#1\}}}
\newcommand{\cmd}[2]{\textrm{\textbackslash #1\{\textlangle #2\textrangle\}}}
\newcommand{\cmdd}[1]{\textrm{\textbackslash #1}}
\newcommand{\ncmd}{\textrm{\textbackslash newcommand\{\textlangle\textbackslash Befehlname\textrangle\}[\textlangle Anzahl Parameter \textrangle]\{\textlangle Befehlsdefinition\textrangle\}}}
\newcommand{\ncmdd}[2]{\textrm{\textbackslash newcommand\{\textbackslash #1\}\{#2\}}}
\newcommand{\dmop}[2]{\textrm{\textbackslash DeclareMathOperator\{\textlangle\textbackslash #1\textrangle\}\{\textlangle #2\textrangle\}}}
\newcommand{\dmopp}[2]{\textrm{\textbackslash DeclareMathOperator\{\textbackslash #1\}\{#2\}}}
\newcommand{\umg}[1]{\textcolor{umg}{#1}}
\newcommand{\pkg}[1]{\textcolor{pkg}{#1}}


\DeclareMathOperator{\ArcSin}{ArcSin}
\DeclareMathOperator{\sgn}{sgn}

\makeatletter
        \providecommand\grpn[1]{\textcolor{OliveGreen}{\texttt{#1}}}
        \providecommand\cls[1]{\textcolor{BurntOrange}{\textsf{#1}}}
        \providecommand\pkg[1]{\cls{#1}}
        \providecommand\umg[1]{\textcolor{SeaGreen!50!Brown}{\texttt{#1}}}
        \providecommand\msur[1]{\texttt{\{#1\}}}
        \providecommand\osur[1]{\texttt{[#1]}}
        \providecommand\meta[1]{\emph{\ensuremath\langle#1\ensuremath\rangle}}
% \cmd{\foo} Prints \foo verbatim
        \def\cmd#1{\cs{\expandafter\cmd@to@cs\string#1}}
        \def\cmd@to@cs#1#2{\char\number`#2\relax}
        \DeclareRobustCommand\cs[1]{\texttt{\char`\\#1}}
% \marg \marg{text} prints {text}, `mandatory argument'.
        \providecommand\marg[1]{%
        {\ttfamily\char`\{}\meta{#1}{\ttfamily\char`\}}}
% \oarg \oarg{text} prints [text], `optional argument'.
        \providecommand\oarg[1]{%
        {\ttfamily[}\meta{#1}{\ttfamily]}}
% \parg \parg{te,xt} prints (te,xt), `picture mode argument'.
        \providecommand\parg[1]{%
        {\ttfamily(}\meta{#1}{\ttfamily)}}

% Keine Abstände vor und nach Codeblöcken
\preto{\@verbatim}{\topsep=1pt \partopsep=1pt }
\makeatother

\newcommand{\uebung}[1]{
\begin{frame}
\frametitle{Zeit zum Ausprobieren}

Die Folien findet ihr hier: \url{\slideurl}

Die Aufgaben hier: \url{\exerciseurl}
\bigskip
\begin{center}
    {\huge
    Bearbeitet jetzt die\\
    Aufgabe #1
    }
\end{center}
\end{frame}}
