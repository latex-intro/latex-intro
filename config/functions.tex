
\newenvironment{fframe}{\begin{frame}[fragile,environment=specialframe]}{\end{frame}}
\newenvironment{codeblock}[1][]{\smallskip\begin{block}{Code: #1}}{\end{block}}

\newcommand{\R}{\mathbb{R}}
\DeclareMathOperator{\ArcSin}{ArcSin}
\DeclareMathOperator{\sgn}{sgn}

\newcommand{\cmd}[1]{\texttt{\textbackslash{#1}}}
\newcommand{\marg}[1]{\texttt{\{}\textlangle\emph{#1}\textrangle\texttt{\}}}
\newcommand{\oarg}[1]{\texttt{[}\textlangle\emph{#1}\textrangle\texttt{]}}
\newcommand{\pkg}[1]{\textcolor{BurntOrange}{#1}}
\newcommand{\umg}[1]{\textcolor{SeaGreen!50!Brown}{#1}}

\providecommand{\slideurl}{http://slides.are.here}
\providecommand{\exerciseurl}{http://exercises.are.here}
\providecommand{\exercisesource}{Die Folien findet ihr hier: \url{\slideurl}

Die Aufgaben hier: \url{\exerciseurl}}


\makeatletter
\newbool{@withsolutions}
\booltrue{@withsolutions}

\newbool{@forcesolutions}
\boolfalse{@forcesolutions}

\newbool{@handoutsolutions}
\boolfalse{@handoutsolutions}

\newcommand{\hidesolutions}{\boolfalse{@withsolutions}}
\newcommand{\forcesolutions}{\booltrue{@forcesolutions}\booltrue{@withsolutions}\booltrue{@handoutsolutions}}
\newcommand{\handoutsolutions}{\booltrue{@handoutsolutions}}
\newcommand{\registerexercise}[1]{\externaldocument[U-]{uebungen/#1}}

% Zeige Lösung, wenn nicht manuell verborgen. Interne Funktion für die Anzeige.
% Lösungen werden im Handout nicht angezeigt, wenn nicht \forcesolution verwendet wird.
\newcommand{\@loesung}[2]{%
\ifbool{@withsolutions}{
\ifbool{@handoutsolutions}{
\begin{frame}
    \frametitle{#1}
    #2
\end{frame}
}{
\begin{frame}<handout:0>
    \frametitle{#1}
    #2
\end{frame}
}}{}}

% Lösungsfolie anzeigen. Kann als optionalen Parameter einen Label,
% zeigt die Folie dann nur, wenn 1) \forcesolution gesetzt ist oder
% 2) Das Label auf dem mit \registerexercise registrierten Übungszettel vorhanden ist.
\newcommand{\loesung}[3][]{%
\ifblank{#1}
    {\@loesung{#2}{#3}}
    {\ifcsundef{r@U-#1}{\ifbool{@forcesolutions}{\@loesung{#2}{#3}}{}}{\@loesung{\ref{U-#1}) #2}{#3}}}}

% Interne Funktion zur Ausgabe der Übungsfolie
\newcommand{\@uebung}[1]{
\begin{frame}
\frametitle{Zeit zum Ausprobieren}

\exercisesource{}
\bigskip
\begin{center}
    {\huge
    Bearbeitet jetzt die\\
    #1
    }
\end{center}
\end{frame}}

% Übungsfolie anzeigen, benötigt einen Label als Parameter.
% Wird nur angezeigt, wenn das Label auf dem mit \registerexercise
% registrierten Übungszettel vorhanden ist oder es die beiden Label
% <label>:first und <label>:last gibt, dann wird ein Bereich von
% Ausgaben angegeben.
\newbool{@showuebung}
\newcommand{\uebung}[1]{
\booltrue{@showuebung}
\ifcsundef{r@U-#1:first}{\boolfalse{@showuebung}}{}
\ifcsundef{r@U-#1:last}{\boolfalse{@showuebung}}{}
\ifbool{@showuebung}{\@uebung{Aufgaben \ref{U-#1:first} - \ref{U-#1:last}}}{\ifcsundef{r@U-#1}{}{\@uebung{Aufgabe \ref{U-#1}}}}
}

\makeatother