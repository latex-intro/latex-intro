% Alles was im Kopf steht, ausser Funktionsdefinitionen. Diese sollten besser in
% eine eigene Datei ausgelagert werden. Fuer kurze Funktionen einfach in
% functions.tex, sonst ggf. fuer zusammenhaengende Dinge eine eigene Datei.

% Titel, Autor etc. sollten gesetzt werden. Man kann sie bei Bedarf einfach in
% der jeweiligen Praesentation ueberschreiben.

\usepackage[ngerman]{babel}
\usepackage[utf8]{inputenc}
\usepackage[T1]{fontenc}
\usepackage{csquotes}

\usepackage{tikz}
% Für lustige Grafiken
\usetikzlibrary{shapes,arrows}
\usetikzlibrary{positioning}
\usetikzlibrary{arrows}
\usetikzlibrary{calc,fadings,decorations.pathreplacing}

\usepackage[european]{circuitikz}

\usepackage{color}
\usepackage{textcomp}

\graphicspath{{./images/}}

\renewcommand{\familydefault}{\sfdefault}
\usetheme[footlinenumber, navline=false, footlineauthor]{Bremen}

\fbnum{00}
\fbname{Uni Bremen}
\subject{\LaTeX}

\title{Eine Einführung in \LaTeX{}}
\author[Arbeitsgruppe \LaTeX]{Arbeitsgruppe \LaTeX}
\date{Orientierungswoche \the\year{}}

%Hinzugefügt von Robin
\usepackage{color}
\usepackage{ textcomp }
\usepackage{mathtools}

\institute[Uni Bremen]{-- Universität Bremen --}

\providecommand{\slideurl}{http://slides.are.here}
\providecommand{\exerciseurl}{http://exercises.are.here}

\makeatletter
\g@addto@macro\beamer@lastminutepatches{\begin{frame}
    \titlepage{}
\end{frame}}
\makeatother

\AtBeginSection{\begin{frame}
    \begin{center}
    \Huge \textbf{\secname}
    \end{center}        
\end{frame}}

\makeatletter
\setbeamertemplate{footline}{%
  \usebeamerfont{subsection in head/foot}%
  \mbox{}\rlap{%
    \begin{pgfpicture}{0pt}{0pt}{\paperwidth}{2.8\baselineskip}%
      % Grauer Fond
      \color{coolgray 1}%
      \pgfpathrectangle{\pgfpoint{0cm}{0cm}}{%
	\pgfpoint{\paperwidth}{1.6\baselineskip}}\pgfusepath{fill}
      \pgfpathrectangle{\pgfpoint{0mm}{0cm}}{%
	\pgfpoint{.75\paperwidth}{2.8\baselineskip}}\pgfusepath{fill}
      \color{black}%
      % Universitaetslogo
%      \pgfputat{\pgfpoint{4mm}{1.5\baselineskip}}{%
%	\pgfbox[left,bottom]{\pgfimage[height=2\baselineskip]{unilogo}}%
      %}
    \end{pgfpicture}%
  }
  %% Universitätslogo
  \rlap{%
    \raisebox{.55\baselineskip}{%
      \mbox{}\hspace{4mm}%
      \parbox[b]{.75\paperwidth}{%
	\href{http://www.uni-bremen.de}{%
	  \includegraphics[height=1.7\baselineskip]{unilogo}}%
	}}}%
  %% Institute/Partnerlogos
  \rlap{%
    \raisebox{1.75\baselineskip}{%
      \mbox{}\hspace{.5\paperwidth}%
      \llap{%
      \parbox[t]{.15\paperwidth}{%
        \vspace{0pt}%
	\mbox{}\hfill%
	\beamer@theme@footline@partnerlogo}%
	\hspace{.025\paperwidth}}
      \rlap{%
      \hspace{.025\paperwidth}%
      \parbox[t]{0.15\paperwidth}{%
        \vspace{0pt}%
	\beamer@theme@footline@secondarypartnerlogo}%
	}}}%
  %% Autor
  \rlap{%
    \raisebox{1.9\baselineskip}{%
      \mbox{}\hspace{.75\paperwidth}%
      \parbox[b]{.25\paperwidth}{%
	\mbox{}\hspace{.5mm}\beamer@theme@footline@author\hfill%
% Ausgelager siehe unten
%	\beamer@theme@footline@number\hfill~\hspace{5mm}
  }}}%
  %%% Foliennummer
  \rlap{%
    \raisebox{1.9\baselineskip}{%
      \mbox{}\hspace{.91\paperwidth}%
      \parbox[b]{.05\paperwidth}{%
        \raggedright\beamer@theme@footline@number
      }
    }
  }
  %% Abschnittsleiste
  \rlap{\raisebox{1.9\baselineskip}{%
      \beamer@theme@footline@section%
  }}%
  %% Navigationssymbole
  \rlap{\parbox[b]{\paperwidth}{%
      \hfill\usebeamertemplate***{navigation symbols lower}%
      \hspace{1.5mm}\mbox{}\vspace{.5mm}%
  }}%
}
\makeatother



\newenvironment{fframe}{\begin{frame}[fragile,environment=specialframe]}{\end{frame}}
\newenvironment{codeblock}[1][]{\smallskip\begin{block}{Code: #1}}{\end{block}}

\definecolor{pkg}{HTML}{FF7F00}
\definecolor{umg}{HTML}{32CD32}

\newcommand{\R}{\mathbb{R}}

\newcommand{\upk}[1]{\textrm{\textbackslash usepackage\{#1\}}}
\newcommand{\cmd}[2]{\textrm{\textbackslash #1\{\textlangle #2\textrangle\}}}
\newcommand{\cmdd}[1]{\textrm{\textbackslash #1}}
\newcommand{\ncmd}[2]{\textrm{\textbackslash newcommand\{\textlangle #1\textrangle\}\{\textlangle #2\textrangle\}}}
\newcommand{\umg}[1]{\textcolor{umg}{#1}}
\newcommand{\pkg}[1]{\textcolor{pkg}{#1}}

