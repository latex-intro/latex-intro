% Alles was im Kopf steht, ausser Funktionsdefinitionen. Diese sollten besser in
% eine eigene Datei ausgelagert werden. Fuer kurze Funktionen einfach in
% functions.tex, sonst ggf. fuer zusammenhaengende Dinge eine eigene Datei.

% Titel, Autor etc. sollten gesetzt werden. Man kann sie bei Bedarf einfach in
% der jeweiligen Praesentation ueberschreiben.

\usepackage[ngerman]{babel}
\usepackage[utf8]{inputenc}
\usepackage[T1]{fontenc}
\usepackage{etoolbox}
\usepackage{csquotes}
\usepackage{amsmath,amssymb,amsthm}
\usepackage{color}
\usepackage{textcomp}
\usepackage{mathtools}
\usepackage[locale=DE]{siunitx}
\usepackage{xr}
\usepackage[version=3,arrows=pgf]{mhchem}
\usepackage{lipsum}
\usepackage{wrapfig}
\usepackage{fancyvrb}

\usepackage{tikz}
% Für lustige Grafiken
\usetikzlibrary{shapes,arrows}
\usetikzlibrary{positioning}
\usetikzlibrary{arrows}
\usetikzlibrary{calc,fadings,decorations.pathreplacing}

\usepackage[european]{circuitikz}

\hypersetup{
    bookmarks=true,
    unicode=false,
    pdfstartview={FitH},
    pdftitle={Eine Einführung in LaTeX},
    pdfauthor={Arbeitsgruppe LaTeX, Universität Bremen},
    hidelinks=true
}

\definecolor{coolgray}{rgb}{0.90625,0.8828125,0.8359375}
\definecolor{BurntOrange}{rgb}{1,0.5,0}
\definecolor{Brown}{HTML}{C17D11}
\definecolor{SeaGreen}{HTML}{2e8b57}

\graphicspath{{./images/}}

\renewcommand{\familydefault}{\sfdefault}

\ifx \customtheme \undefined \usetheme{metropolis}
\metroset{block=fill}
\setbeamercolor{framesubtitle}{fg=mDarkTeal}
\addtobeamertemplate{frametitle}{}{%
  \ifx\insertframesubtitle\@empty\else%
  \usebeamerfont{framesubtitle}%
  \usebeamercolor[fg]{framesubtitle}%
  \vspace{1em}\insertframesubtitle%
  \fi%
}
\newcommand{\fbnum}[1]{}
\newcommand{\fbname}[1]{}
\else%
\input{config/themes/\customtheme}
\fi%

\subject{\LaTeX}

\title{Eine Einführung in \LaTeX{}}
\author[AG \LaTeX]{Arbeitsgruppe \LaTeX\\\smallskip
\footnotesize{}Robin Görmer, Malte Gerken\\
Yannik Schädler}
\date{Orientierungswoche \the\year{}}


\institute[Uni Bremen]{-- Universität Bremen --}

\makeatletter
% Titelseite automatisch erzeugen
\g@addto@macro\beamer@lastminutepatches{\begin{frame}
    \titlepage{}
\end{frame}}

% Keine Abstände vor und nach Codeblöcken
\preto{\@verbatim}{\topsep=1pt \partopsep=1pt }
\makeatother


\newenvironment{fframe}{\begin{frame}[fragile,environment=specialframe]}{\end{frame}}


\makeatletter
        \providecommand\grpn[1]{\textcolor{OliveGreen}{\texttt{#1}}}
        \providecommand\cls[1]{\textcolor{BurntOrange}{\textsf{#1}}}
        \providecommand\pkg[1]{\cls{#1}}
        \providecommand\umg[1]{\textcolor{SeaGreen!50!Brown}{\texttt{#1}}}
        \providecommand\msur[1]{\texttt{\{#1\}}}
        \providecommand\osur[1]{\texttt{[#1]}}
        \providecommand\meta[1]{\emph{\ensuremath\langle#1\ensuremath\rangle}}
% \cmd{\foo} Prints \foo verbatim
        \def\cmd#1{\cs{\expandafter\cmd@to@cs\string#1}}
        \def\cmd@to@cs#1#2{\char\number`#2\relax}
        \DeclareRobustCommand\cs[1]{\texttt{\char`\\#1}}
% \marg \marg{text} prints {text}, `mandatory argument'.
        \providecommand\marg[1]{%
        {\ttfamily\char`\{}\meta{#1}{\ttfamily\char`\}}}
% \oarg \oarg{text} prints [text], `optional argument'.
        \providecommand\oarg[1]{%
        {\ttfamily[}\meta{#1}{\ttfamily]}}
% \parg \parg{te,xt} prints (te,xt), `picture mode argument'.
        \providecommand\parg[1]{%
        {\ttfamily(}\meta{#1}{\ttfamily)}}

% Keine Abstände vor und nach Codeblöcken
\preto{\@verbatim}{\topsep=1pt \partopsep=1pt }
\makeatother

\newcommand{\uebung}[1]{
\begin{frame}
\frametitle{Zeit zum Ausprobieren}

Die Folien findet ihr hier: \url{\slideurl}

Die Aufgaben hier: \url{\exerciseurl}
\bigskip
\begin{center}
    {\huge
    Bearbeitet jetzt die\\
    Aufgabe #1
    }
\end{center}
\end{frame}}
