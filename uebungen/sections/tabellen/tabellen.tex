\item Erstelle eine 3-spaltige Tabelle mit den ersten 5 natürlichen Zahlen, ihren Quadraten und ihren dritten Potenzen. 

\begin{loesung}
z.B. 
\verb|\begin{tabular}{ccc}|
\verb| 1 & 1 & 1 \\|
\verb|...|
\verb|\end{tabular}|
\end{loesung}

\item Mache die erste Spalte linksbündig, die zweite rechtsbündig, die dritte exakt $3.14\,$cm breit. Benutze 
die maximal mögliche Anzahl von Linien (zwischen Spalten und Zeilen und an den Rändern)

\begin{loesung}
z.B. 
\begin{verbatim}
\begin{tabular}{|l|r|p{3.14cm}|} \hline
 1 & 1 & 1 \\ \hline
...
\end{tabular}
\end{verbatim}
\end{loesung}



\item Benutze das \texttt{array}-Paket, um die erste Spalte der Tabelle fett zu formatieren. Füge 2 Zellen zusammen zu einer langen Zelle.


\begin{loesung}
z.B. 
\begin{verbatim}
\usepackage{array}
\begin{tabular}{|l >{\bfseries}|r|p{3.14cm}|} \hline
 1 & 1 & 1 \\ \hline
...
5 & \multicolumn{2}{c}{vereinigte Zelle}
\end{tabular}
\end{verbatim}
\end{loesung}

\item benutze das Pakte \texttt{longtable}, um eine seitenübergreifende Tabelle zu setzen. Schreibe über die 2.Tabellenhälfte einen kurzen Kommentar 
\emph{hier geht die Tabelle weiter}


\begin{loesung}
z.B. 
\begin{verbatim}
\usepackage{longtable}
 
\begin{longtable}{|c|c|}
\hline Zeit t & Geschwindigkeit $v_{\text{B}}$  \\ \hline 
\endfirsthead

\multicolumn{2}{c}{Hier geht die Tabelle weiter} \\ \hline 
Zeit t & Geschwindigkeit $v_{\text{B}}$  \\ \hline 
\endhead

\hline \multicolumn{2}{|r|}{{Weiter auf der nächsten Seite}} \\ 
\endfoot

1 & 2 \\ 
 ...
\end{longtable}
\end{verbatim}
\end{loesung}

