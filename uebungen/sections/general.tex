\section{Dokumentenformatierungen}
\begin{uebung}
\item Erstelle ein Minimaldokument mit dem Code aus 
	Abbildung~\ref{minimaldoc} und erstelle daraus eine PDF.\label{firststart:first}
\item Teste was passiert, wenn du die \pkg{inputenc}-Zeile auskommentierst.
    Probiere was passiert, wenn du anstatt \texttt{utf8} zum Beispiel
    \texttt{latin1} verwendest. Wo ist der Unterschied?\label{inputenc}
\item Versuche ein paar Sonderzeichen zu schreiben: \textbackslash{} und \^{}Hallo\label{specialchars}
\item Füge deiner Datei mehr Text hinzu; verwende dazu das Paket 
	\pkg{lipsum} und den Befehl \verb+\lipsum[1-7]+. Dies fügt
	7 Absätze eines lateinischen Fülltextes (\enquote{Lorem ipsum}) ein.
	\begin{hinweis}
	    Pakete werden in der Präambel mit \cmd{usepackage}\marg{Paketname} geladen.
	\end{hinweis}\label{firststart:last}
\item Füge deiner Datei \emph{Titel}, \emph{Autor} und 
	\emph{Datum} und stelle diese Angaben im Titel dar.
	Füge außerdem ein \emph{Inhaltsverzeichnis} hinzu.
	\begin{hinweis}
	    Du musst \LaTeX\ evtl. mehrmals ausführen, damit alle 
		Seitenzahlen korrekt sind.
	\end{hinweis}\label{markup:first}
\item Füge nun noch 4-5 Absätze vor der letzten Überschrift ein.
    Schaue nach, ob sich die Seitenzahlen im Inhaltsverzeichnis geändert haben.
    Was musst du tun um die Seitenzahlen zu korrigieren?
\item Ersetze die Option \texttt{ngerman} vom \pkg{babel}-Paket durch
	\texttt{english} und finde heraus was passiert.
\item Tausche an beliebiger Stelle ein \cmd{section}\marg{Text} gegen 
	\cmd{section*}\marg{Text} aus und beobachte das Verhalten von \LaTeX; 
	beobachte was passiert, wenn du \cmd{section}\oarg{Text1}\marg{Text2} 
	anstelle der Variante mit Sternchen verwendest.
	\label{aufg:sectionstar}

\begin{figure}[h]
	\begin{minipage}[t]{.5\textwidth}
	\caption{Minimaldokument}\label{minimaldoc}
	\begin{verbatim}
\documentclass{scrartcl}
\usepackage[utf8]{inputenc}
\usepackage[T1]{fontenc}
\usepackage[ngerman]{babel}
\begin{document}
\section{Überschrift}
Etwas Text...
\subsection{Unter-Überschrift}
Noch mehr Text...
\subsubsection{UU-Überschrift}
Lorem ipsum dolor sit amet,\ldots
\end{document}
	\end{verbatim}
	\end{minipage}
	\begin{minipage}[t]{.5\textwidth}
	\caption{Liste zum Nachbauen}\label{enumerate}\small
	\begin{enumerate}
		\item Hier geht es um Listen
		\item Listen kann man schön verschachteln
		\begin{itemize}
			\item man muss dabei einiges beachten
			\item \LaTeX\ ist eine strukturierte Sprache
			\item[!] Zeichen lassen sich ändern
			\begin{enumerate} 
				\item Verschachtelung geht aber immer
				\item \dots\ wie man hier sieht!
			\end{enumerate}
		\end{itemize}
		\item man muss sich beim Eingeben nur konzentrieren, sonst 
			passieren Fehler:
		\begin{enumerate}
			\item beliebt ist, eine \umg{enumerate}"=Umgebung mit 
				einer \umg{itemize}"=Umgebung zu schließen
			\item oder statt \texttt{\{itemi\textit{z}e\}} 
				\texttt{\{itemi\textit{s}e\}} zu tippen
		\end{enumerate}
	\end{enumerate}
	\end{minipage}
\end{figure}
\clearpage

\item Experimentiere mit den Zählern \texttt{secnumdepth} und 
	\texttt{tocdepth}, dies geschieht in der Präambel (oder Kopf) der Datei mit dem
	Befehl \cmd{setcounter}\marg{Zähler}\marg{Wert}. \label{zaehler}
\item Probiere verschiedene Textausrichtungen aus: Füge am Ende
    deines Dokumentes jeweils einen Absatz \enquote{Lorem Ipsum}
    linksbündig und zentriert ein. Baue außerdem an einer beliebigen Stelle
    im Dokument einen Absatz \enquote{Lorem Ipsum} als Zitat ein.
\item Füge nun an die Überschriften ein Label an (mittels \cmd{label}).
    Verweise dann irgendwo anders mit \cmd{ref} und auch
    \cmd{pageref} auf diese Label. Probiere dies auch mit einer \enquote{gesternten}
    Section Variante aus Aufgabe \ref{aufg:sectionstar}.
\item Füge nun die Liste aus Abbildung \ref{enumerate} in dein Dokument ein.
\item Labels kann man auch mit Listen verwenden. Füge deiner Liste ein Label
    hinzu und Verweise an anderer Stelle im Dokument darauf.
\item Experimentiere nun mit den Dokumentklassen. Du solltest auf jeden
    Fall einmal \texttt{scrbook} und \texttt{scrreprt} ausprobieren.
    
    Nun probiere außerdem die folgenden Dinge:
    \begin{enumerate}
        \item die Papiergröße auf DIN A5 zu ändern
        \item zweispaltigen Text zu setzen
        \item den Einzug am Anfang der Zeile eines neuen Absatzes zu
            Entfernen (Versuche die Option \texttt{parskip})
        \item Probiere einmal Querformat
        \item Kannst du den Text mit der Buchklasse auch einseitig setzen?
    \end{enumerate}
\item Füge deinem Dokument einen Anhang hinzu. Im Anhang sollte mindestens
    eine Überschrift und ein Absatz Text sein.
    \begin{hinweis}
        Den Anhang beginnst du mittels \cmd{appendix}. Was ändert sich
        danach an den Überschriften? Schaue auch im Inhaltsverzeichnis nach.
    \end{hinweis}\label{markup:last}
\end{uebung}