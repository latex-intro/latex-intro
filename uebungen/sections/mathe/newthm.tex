\item 
    Befasst euch näher mit dem \pkg{amsthm}-Paket. Wie könnt ihr den Stil eines
    eigenen Theorems ändern? Wie könnt ihr die Nummerierung von Gleichungen und
    Sätzen auf einen Abschnitt beschränken? 
    \begin{loesung}
        Der Befehl \cmd{newtheoremstyle} ermöglicht das erstellen eines eigenen
        Stils für eigene Theorem-Umgebungen. Er funktioniert wie folgt:
\begin{verbatim}
\newtheoremstyle{others} 
  {\baselineskip}  % ABOVESPACE
  {\baselineskip}  % BELOWSPACE
  {\itshape}       % BODYFONT
  {}               % INDENT (empty value is the same as 0pt)
  {\bfseries}      % HEADFONT
  {:}              % HEADPUNCT
  {.5em}           % HEADSPACE. 
  {}               % CUSTOM-HEAD-SPEC
\end{verbatim}
        Habt ihr einen eigenen Stil definiert, könnt ihr nun nach dem Befehl \\
        \cmd{theoremstyle}\marg{Stil} definieren, welcher Umgebung diesem Stil
        folgen soll. Das sieht dann wie folgt aus:
\begin{verbatim}
\theoremstyle{others}
\newtheorem{mythm2}{Theorem}[section]
\end{verbatim}
        Mit der Option \verb|[section]| wird nun jedes Theorem innerhalb einer
        Section durchnummeriert. 
                    
        Gleichungen innerhalb einer Umgebung durchzunummerieren erhaltet ihr
        durch den Befehl \cmd{numberwithin}\marg{Umgebung}{Gliederung}. Dabei sind
        equations, also Gleichungen durchnummerierte Umgebungen!
    \end{loesung}
              
        
       
