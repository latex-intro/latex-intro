\section{Formelsatz}
\begin{uebung}

\item Finde die \LaTeX{} Kommandos für $\forall\,\oslash\,\leftrightsquigarrow\,\otimes\,\uplus\,\hat{a}$.
        \begin{loesung}
            \verb|\forall \oslash \leftrightsquigarrow \otimes \uplus \hat{abc}|
        \end{loesung}
\item Erstelle mit \cmd{newcommand} neue Kommandos, um den Schreibaufwand zu reduzieren.
        \begin{enumerate}[label=(\alph*)]
            \item \cmd{H} soll das Kommando für $\mathbb{H}$ sein und \cmd{eps} soll $\varepsilon$ ausgeben.
            \item \cmd{xvec}\{n\} soll $(x_1,\dots,x_n)$ und \cmd{xVec}\{n\} soll 
                \[
                    \begin{pmatrix}
                    x_1 \\ \vdots \\ x_n
                    \end{pmatrix}
                \]
                ausgeben. Die Punkte $\dots$ sollen dabei nicht ersetzt werden. 
            \item \cmd{abs}\{x\} soll $\abs{x}$ ausgeben und \cmd{norm}\{x\} soll $\norm{x}$ ausgeben. 
            \item \cmd{i} soll $\mathfrak{i}$ ausgeben \cmd{complex}\{z\} soll $\complex{z}$ ausgeben.
        \end{enumerate}
        \begin{loesung}
            \begin{enumerate}[label=(\alph*)]
            \item \verb|\newcommand{\H}{\mathbb{H}}| und \verb|\newcommand{\eps}{\varepsilon}|
            \item \verb|\newcommand{\xvec}[1]{(x_1,\dots,x_{#1}}|; Hier geht natürlich auch die \umg{pmatrix}-Umgebung. Weiter \\
            \begin{verbatim}
            \newcommand{xVec}[1]{
            \begin{pmatrix}
            x_1\\ \vdots\\x_{#1}
            \end{pmatrix}}
            \end{verbatim}
            \item \begin{verbatim}
                  \newcommand{\abs}[1]{\lvert#1\rvert}
                  \newcommand{\norm}[1]{\lVert#1\rVert}
                  \end{verbatim}
            \item Dieses Beispiel ist besonders interessant, denn \cmd{i} ist bereits vordefiniert: \\
                    \begin{verbatim}
                    \renewcommand{\i}{\mathfrak{i}}
                    \newcommand{\complex}[1]{
                      \re\left(#1\right)+\i\im\left(#1\right)}
                    \DeclareMathOperator{\re}{Re}
                    \DeclareMathOperator{\im}{Im}
                    \end{verbatim}
            \end{enumerate}
        \end{loesung}
\item Setze die folgenden Texte in \LaTeX{} um:
        \begin{enumerate}[label=(\alph*)]
         \item Seien $a,b,c\in\R$, mit $a,b$ die Katheten und $c$ die Hypotenuse eines rehtwinkligen Dreiecks. Dann gilt: 
                \[
                    a^2+b^2=c^2.
                \]
          \item Die Summenformel für die geometrische Folge lautet:
          \[
              s_n = 1+ q+q^2+\dots+q^n = \sum_{k=0}^{n}q^k = \frac{1-q^{n+1}}{1-q}
          \]
          Für den Grenzwert von $s_n$ gilt für $\abs{q}<1$: 
          \[
              s_n\xrightarrow{n\rightarrow\infty} \frac{1}{1-q}
          \]
          \item  
              \begin{mythm}[Existenz des linksneutralen Elements]
                  Sei $(G,\circ)$ eine Gruppe.
                  \begin{enumerate}[label=(\roman*)]
                      \item Sind $e_1$ und $e_2$ neutrale Elemente, so folgt $e_1=e_2$.
                      \item Sind $h_1$ und $h_2$ beide invers zu $g\in G$, so folgt $h_1 = h_2$.  
                  \end{enumerate}
              \end{mythm}
                  \begin{proof}\text{ }\\
                  \begin{enumerate}[label=(\roman*)]
                      \item Ausnutzung der Eigenschaft des neutralen Elements liefert $e_1 = e_1\circ e_2=e_2$.
                      \item Es gilt: 
                          \begin{alignat*}{2}
                              h_2 &= e\circ h_2 = (h_1\circ g)\circ h_2\\
                                  &= h_1 \circ (g\circ h_2)\\
                                  &= h_1 \circ e = h_1
                          \end{alignat*}  
                  \end{enumerate}
                  \end{proof}
                  \begin{hinweis}
                      Schaut euch das \pkg{amsmath}-Paket an.
                  \end{hinweis}
          \item 
              Befasst euch näher mit dem \pkg{amsthm}-Paket. Wie könnt ihr den Stil eines eigenen Theorems ändern? Wie könnt ihr die Nummerierung von Gleichungen und Sätzen auf einen Abschnitt beschränken? 
              
        \end{enumerate}
        \begin{loesung}
            \begin{enumerate}
                \item Das Element-Zeichen erhalten wir durch den Befehel \cmd{in}. Zum Schluss wird der Abgesetzte Mathe-Modus gestartet mit \cmd{[} \dots \cmd{]}. 
                \item Die Punkte $\dots$ erhalten wir wie gewohnt durch \cmd{dots}. Im abgesetzten Mathe-Modus erhalten wir die Summe $\sum$ durch \cmd{sum} und über \\ \verb|\sum^{Text}_{Text}| erhalten wir die Grenzwerte. Die Brüche werden über \cmd{frac}\marg{Zähler}\marg{Nenner} eingeführt. Der Pfeil, der hier benutzt werden soll ist \cmd{xrightarrow}\marg{text}.
                \item Das $\circ$ erhalten wir durch \cmd{circ}. Die \umg{enumerate}-Umgebung wird wie gewohnt mit \verb|[label=(\roman*)]| in der Nummerierung angepasst. Wir haben hier ein eigenes Theorem über das \pkg{amsthm}-Paket geladen und über den Befehl \cmd{newtheorem}\marg{Name}\oarg{Zählung}\marg{Bezeichnung}\oarg{Gliederung} das neue Theorem definiert. Im header sollte dann \verb|\newtheorem{mythm}{Satz}| o.Ä. stehen, sodass ihr die \umg{mythm}-Umgebung erzeugt habt. Was folgt ist eine \umg{proof}-Umgebung, die automatisch den Kasten am Ende und \glqq Beweis:\grqq{} setzt.
                \item Der Befehl \cmd{newtheoremstyle} ermöglicht das erstellen eines eigenen Stils für eigene Theorem-Umgebungen. Er funktioniert wie folgt:\\
\begin{verbatim}
\newtheoremstyle{others} 
  {\baselineskip}  % ABOVESPACE
  {\baselineskip}  % BELOWSPACE
  {\itshape}       % BODYFONT
  {}               % INDENT (empty value is the same as 0pt)
  {\bfseries}      % HEADFONT
  {:}              % HEADPUNCT
  {.5em}           % HEADSPACE. 
  {}               % CUSTOM-HEAD-SPEC
\end{verbatim}
                    Habt ihr einen eigenen Stil definiert, könnt ihr nun nach dem Befehl \\ \cmd{theoremstyle}\marg{Stil} definieren, welcher Umgebung diesem Stil folgen soll. Das sieht dann wie folgt aus:
\begin{verbatim}
\theoremstyle{others}
\newtheorem{mythm2}{Theorem}[section]
\end{verbatim}
                    Mit der Option \verb|[section]| wird nun jedes Theorem innerhalb einer section durchnummeriert. 
                    
                    Gleichungen innerhalb einer Umgebung durchzunummerieren erhaltet ihr durch den Befehl \cmd{numberwithin}\marg{Umgebung}{Gliederung}. Dabei sind equations, also Gleichungen durchnummerierte Umgebungen!
            \end{enumerate}
        \end{loesung}
\end{uebung}