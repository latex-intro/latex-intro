\section{Weitere Aufgaben}
    \begin{uebung}
        \item Du kannst dein Dokument in mehrere Dateien aufteilen, das ist vor
        allem bei langen Dokumenten sinnvoll. Verschiebe die komplette Präambel
        (außer die \cmd{documentclass}-Zeile) in die Datei \texttt{Kopf.tex} und
        die letzten Zwei Kapitel vor dem Anhang in die Datei \texttt{Schluss.tex}.
        
        In deinem Hauptdokument kannst du die Dateien jetzt mit
        \cmd{input}\marg{Dateiname} einbinden. Was ist der Unterschied zu
        \cmd{include}\marg{Dateiname}?
            \begin{hinweis}
                Das Erstellen des Dokuments funktioniert dann nur noch aus der Hauptdatei.
                Um das zu ändern, wechsle zur Hauptdatei und klicke auf
                \enquote{Optionen \textrightarrow{} Aktuelle Datei zur Masterdatei erklären}.
            \end{hinweis}
            \begin{loesung}
                \cmd{include} macht unter anderem ein \cmd{clearpage} vor und nach
                der Datei. Siehe auch \href{http://tex.stackexchange.com/questions/246/when-should-i-use-input-vs-include}{diesen Stack Exchange Thread}.
            \end{loesung}
    \end{uebung}