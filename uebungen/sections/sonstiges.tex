\section{Weitere Aufgaben}
    \begin{uebung}
        \item Du kannst dein Dokument in mehrere Dateien aufteilen, das ist vor
        allem bei langen Dokumenten sinnvoll. Verschiebe die komplette Präambel
        (außer die \cmd{documentclass}-Zeile) in die Datei \texttt{Kopf.tex} und
        die letzten Zwei Kapitel vor dem Anhang in die Datei \texttt{Schluss.tex}.

        In deinem Hauptdokument kannst du die Dateien jetzt mit
        \cmd{input}\marg{Dateiname} einbinden. Was ist der Unterschied zu
        \cmd{include}\marg{Dateiname}?
            \begin{hinweis}
                Das Erstellen des Dokuments funktioniert dann nur noch aus der Hauptdatei.
                Um das zu ändern, wechsle zur Hauptdatei und klicke auf
                \enquote{Optionen \textrightarrow{} Aktuelle Datei zur Masterdatei erklären}.
            \end{hinweis}
            \begin{loesung}
                \cmd{include} macht unter anderem ein \cmd{clearpage} vor und nach
                der Datei. Siehe auch \href{http://tex.stackexchange.com/questions/246/when-should-i-use-input-vs-include}{diesen Stack Exchange Thread}.
            \end{loesung}

        \item Erstelle eine Tabelle, in welcher du ein paar Messwerte darstellst.
            Verwende hierzu einfach ein paar Zahlen mit Nachkommastellen und Einheiten.
            Setze die Einheiten korrekt und richte die Zahlen am Komma aus.
            \begin{hinweis}
                Praktisch vor allem für Praktikumsprotokolle, das Paket \pkg{siunitx} ist
                hier hilfreich!
            \end{hinweis}

        \item Füge deinem Dokument noch ein paar Formeln hinzu, welche nummeriert werden
            sollen. Erstelle außerdem einige Verweise auf die Formeln. Mit dem Paket
            \pkg{mathtools} kann da Nummerierung bei nicht referenzierten Formeln
            automatisch ausgeblendet werden (Option \verb|showonlyrefs|). Recherchiere
            die Verwendungsweise und eventuell auftretende Probleme und wende die Option an.

        \item Man kann die Positionierung von \umg{figure} und \umg{table} Umgebungen mit
            Angaben wie \verb|[tp]| beeinflussen. Dafür muss man aber an jedem Gleitobjekt
            diese Angabe machen. Unter \url{https://www.latex-project.org/publications/tb111mitt-float.pdf}
            findest du eine sehr gute Übersicht, wie man die Positionierung beeinflussen kann.

            Wie schafft man es, dass \LaTeX{} bis zu 80\%{} der Seite mit Gleitobjekten füllt,
            die mit \verb|[b]| positioniert werden sollen?
            Erstelle hierzu einige Gleitobjekte und teste das Verhalten.
    \end{uebung}
