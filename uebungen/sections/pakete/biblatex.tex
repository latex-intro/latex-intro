\item Erstelle eine Bibliographie mit Biblatex. Gehe dazu wie folgt vor:
	\begin{figure}[bp!]
	\lstinputlisting[caption={Beispielhafte Bibliographie-Datei},label=bibfile]{files/bibliography.bib}
	\end{figure}
    \begin{enumerate}
        \item Zuerst brauchst du eine Bibliographie-Datei. In dieser sind die 
            jeweiligen Quellen eingetragen. Die einzelnen Einträge beginnen immer
            mit einem @, gefolgt von dem Typ der Quelle. Direkt danach steht der
            Zitierschlüssel, welchen du beim Verweisen benötigst.
        
            Für den Anfang nutze die Beispieldaten aus dem Listing \ref{bibfile}.
            Erstelle eine neue Datei, kopiere den Inhalt des Listings herein und
            speichere sie unter dem Namen \texttt{Literatur.bib} ab.
        
        \item Lade nun das Paket \pkg{biblatex} mit den Optionen
            \texttt{backend=biber} und \texttt{style=authoryear}. Außerdem musst
            du \LaTeX{} sagen, wo deine Literaturdatenbank zu finden ist, dies
            machst du mit dem Befehl \cmd{addbibresource}\marg{Dateiname}
            \begin{loesung}
                \begin{verbatim}
\usepackage[backend=biber,style=authoryear]{biblatex}
\addbibresource{Literatur.bib}
                \end{verbatim}
            \end{loesung}
        
        \item Jetzt kannst du auf eine der Quellen verweisen. Probiere das mit
            zwei der drei Quellen aus.
            \begin{hinweis}
                Beispiel für \enquote{Per Anhalter durch die Galaxis}:
                \cite{Adam2004}.
            
                Wenn die Ausgabe nur den verwendeten Schlüssel in Fett anzeigt,
                benutzt du kein Latexmk. Wenn möglich wechsle zu Latexmk, da
                dieses Programm \texttt{biber} aufruft. Wenn das nicht möglich ist,
                gehe zu \enquote{Benutzer/in \textrightarrow{} Eigene Befehle
                \textrightarrow{} Eigene Befehle editieren} und füge bei einem
                Eintrag den Befehl \verb|biber %| hinzu. Vergib außerdem einen
                geeigneten Namen. Jetzt kannst du Biber aus dem Menü
                \enquote{Benutzer/in \textrightarrow{} Eigene Befehle} aufrufen.
            \end{hinweis}
            \begin{loesung}
                Mit \cmd{cite}\marg{Schlüssel}
            \end{loesung}
        
        \item Für ein Literaturverzeichnis in den Anhang ein.
            \begin{loesung}
                Mit \cmd{printbibliography}
            \end{loesung}
        
        \item Füge den dritten Eintrag ins Literaturverzeichnis hinzu, ohne ihn
            sichtbar zu referenzieren.
            \begin{loesung}
                Mit \cmd{nocite}\marg{Schlüssel}
            \end{loesung}
        
        \item Füge eine Paketoption hinzu, um die ISBN zu verbergen.
            \begin{loesung}
                Option: \verb|isbn=false|
            \end{loesung}
        
        \item Wenn du das noch nicht machst, ändere deine Zitierbefehle auf den
            automatischen Modus. Teste dann verschiedene Zitierstile, mindestens
            jedoch \texttt{verbose} und \texttt{numeric}.
            \begin{loesung}
                Mit \cmd{autocite}\marg{Schlüssel}
            \end{loesung}
        
        \item Gehe nun auf die \href{http://www.suub.uni-bremen.de/}{Seite der
            Uni-Bibliothek} und suche nach dem Buch \enquote{LaTeX Referenz}
            von Herbert Voß. Exportiere die Bibliographie-Informationen über den
            kleinen Knopf rechts neben dem Bild und füge sie deiner
            \texttt{Literatur.bib} hinzu. Referenziere das Buch nun an ein oder 
            zwei Stellen im Dokument.
            \begin{hinweis}
                Den vorgegebenen Schlüssel kannst du natürlich ändern.
            \end{hinweis}
        
        \item Warum nutzen wir hier das Backend \emph{biber} anstelle von
            \emph{BibTex}?
            \begin{loesung}
                BibTex unterstützt kein UTF8 und damit keine Umlaute.
            \end{loesung}
    \end{enumerate}
