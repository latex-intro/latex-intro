\documentclass[parskip=half-,DIV14,10pt]{scrartcl}
\usepackage[utf8]{inputenc}
\usepackage[T1]{fontenc}
\usepackage[ngerman]{babel}
%\usepackage{lmodern}

\usepackage{asnment}
\usepackage{soerenmath}

\usepackage{mathtools}
\usepackage{microtype}
\usepackage{hyperref}

% Veranstaltung, Zettel und Abgabe bei jedem Zettel anpassen
\newcommand{\coursename}{Semilin.\ Algebra~0}
\newcommand{\asnmentname}{Übungsblatt~42}
\newcommand{\deadline}{11.\,10.\,2015}

\newcommand{\tutorpar}%
           {Tutor:\\Blaise Pascal}
\newcommand{\shorttutorpar}{\tutorpar}
\newcommand{\studentpar}%
           {Bearbeitet von:\\[1ex]
            Sir Isaac Newton\\[0.3ex]
            Gottfried Wilhelm Leibniz}
\newcommand{\shortstudentpar}%
           {Isaac Newton\\G.\,W.\ Leibniz}
\newcommand{\titlepar}%
           {\bfseries\huge Semilineare\\[-2mm] Algebra 0\\[2mm]
            \mdseries\LARGE\asnmentname\\[2mm]
            \normalsize Abgabe: \deadline}

% Punkte bei jedem Zettel anpassen
\newcommand{\pointtab}{%
  \begin{pointlist}
    1ai:5, 1aii:3, 2:10,
    $\sum$ : +
  \end{pointlist}%
}

\begin{document}

\asnmenttitle % setzt den Titel

Übungszettel in \LaTeX\ abzugeben ist nicht nur eine sehr gute Übung
für spätere Seminar- und Abschlussarbeiten, sondern es bringt noch
eine Reihe weiterer Vorteile.  Der Quelltext lässt sich wunderbar in
Repositories verwalten, von vielen Leuten gleichzeitig bearbeiten, und
man kann in der PDF-"`Reinschrift"' viel besser Fehler korrigieren als
auf dem Papier.  Die Tatsache, dass Formeln in \LaTeX\ vom Text
"`umschlossen"' werden, motiviert dazu, ausführlichere Erklärungen zu
schreiben.

Die Klassen aus dem KOMA-Paket sind zwar im Prinzip sehr gut,
aber nicht auf die Bearbeitung von Übungszetteln ausgelegt.
Dies behebt diese Vorlage.  Sie enthält alle Elemente, die
typischerweise für Übungszettel relevant sind, und bringt sie kompakt
unter.

Am Anfang stehen ein paar Beispiele, wie die Bearbeitung einiger
hypothetischer Übungsaufgaben aussieht, und im Anhang folgt dann die
Erläuterung.

In Informatik, Numerik und angewandten Fächern ist es häufig nötig,
Programmcode mit abzudrucken.  Das hierzu häufig verwendete
\lstinline!listings!-Paket ist hier so konfiguriert, dass es sich in
das Layout einpasst.  Sollte im Rahmen der Veranstaltung bereits eine
andere Vorlage zur Verfügung gestellt werden, sprecht bitte vorher ab,
welche ihr verwendet.

\section{}
\subsection{}
\subsubsection{}

\begin{given}
Sei $A\in\R^{n\times n}$ eine reelle symmetrische Matrix.  Seien
$\lambda,\mu\in\R$ Eigenwerte von $A$ mit zugehörigen Eigenvektoren 
$v,w\in\R^n\setminus\{0\}$, sodass gilt:
\[ Av=\lambda v,\qquad Aw=\mu w. \]
\end{given}

\begin{toshow}
Für $\lambda\neq\mu$ sind $v,w$ orthogonal, also $\iprod{v,w}=0$.
\end{toshow}

\begin{proof}
Betrachte die beiden folgenden Ausdrücke:
\begin{align}
\label{eq:umf1}
\iprod{Av,w}&=\iprod{\lambda v,w}=\lambda\iprod{v,w}\\[3pt]
\label{eq:umf2}
\iprod{v,Aw}&=\iprod{v,\mu w}=\mu\iprod{v,w}
\end{align}

Da $A$ symmetrisch ist, gilt $\iprod{Av,w}=\iprod{v,Aw}$.  Durch
Einsetzen in \eqref{eq:umf1},\eqref{eq:umf2} erhalten wir:
\[
\lambda\iprod{v,w}=\mu\iprod{v,w},\qquad
\text{also}\quad
(\lambda-\mu)\,\iprod{v,w}=0.
\]
Für $\lambda\neq\mu$ folgt daraus $\iprod{v,w}=0$.
\end{proof}

\subsubsection{}

\begin{toshow}
Sei $\lambda\in\C$ ein Eigenwert der hermiteschen Matrix
$A\in\C^{n\times n}$.  Dann ist $\lambda$ rein reell, also
$\Im(\lambda)=0$.
\end{toshow}

\begin{proof}
Betrachte den zugehörigen Eigenvektor $v\in\C^n\setminus\{0\}$.  Weil
$A$ hermitesch und das Skalarprodukts sesquilinear ist, gilt:
\[
\lambda\iprod{v,v}=\iprod{\lambda v,v}=\iprod{Av,v}=\iprod{v,Av}
=\iprod{v,\lambda v}=\overline{\lambda}\iprod{v,v}
\]
Wegen $\iprod{v,v}>0$ ist dann $\lambda=\overline{\lambda}$, also
$\Im(\lambda)=0$.
\end{proof}


\section{}

Die \lstinline!main!-Methode gibt den String \lstinline!"hello, world"! aus.
\begin{lstlisting}[language=java]
/* Kommentare und Strings dürfen auch Umlaute enthalten, wenn man UTF-8 benutzt.
 */
public class Hello {
    public static void main(String[] args) {
        System.out.println("hello, world");
    }
}
\end{lstlisting}

Im folgenden Programm ist das \emph{Heron-Verfahren} implementiert,
welches nach der Iterationsvorschrift
\[
y_{k+1} = \frac{y_k+x/y_k}{2},\qquad x>0
\]
einen Näherungswert für $\sqrt{x}$ liefert.  Der Grenzwert
\[
\lim_{k\to\infty}y_k = \sqrt{x}\qquad\text{für}~y_0>0
\]
wurde in der Vorlesung bewiesen.
\lstinputlisting[language=java]{Wurzel.java}

\appendix

\section{\TeX{}nisches}

Die meisten Dinge, die für jeden Übungszettel neu angepasst werden
müssen (insbesondere Nummer, Abgabefrist und Punktetabelle --
vgl.\ \ref{sec:tabelle}) sind in der \lstinline!.tex!-Datei zu
finden.  Die Datei \lstinline!soerenmath.sty! enthält einige nützliche
Makros, die Tipparbeit ersparen und der Struktur der Lösung dienen
sollen; das ist aber zum großen Teil Geschmackssache.  Ebenfalls ist
hier noch das \lstinline!listings!-Paket konfiguriert, sodass Zeilen
einer Länge von 80~Zeichen gerade ohne Umbruch abgedruckt werden
können.

Die Datei \lstinline!asnment.sty! enthält das Layout.  In den meisten
Situationen sollte es ausreichen, die Variablen in der
\lstinline!.tex!-Datei anzupassen.  Die Makros
\lstinline!\thesection!, \lstinline!\thesubsection! und
\lstinline!\thesubsubsection! sind so ausgelegt, dass zunächst
arabisch, dann alphabetisch und zuletzt römisch nummeriert wird.
Sollte dies in einer Veranstaltung anders gehandhabt werden, können
die entsprechenden Definitionen leicht angepasst werden.

Die Datei \lstinline!cm-super-t1.enc! hat den einzigen Zweck, dass der
Buchstabe "`ß"' besser aussieht.

\subsection{Punktetabelle}
\label{sec:tabelle}

Eine Punktetabelle zu erzeugen ist mit der vorliegenden Umgebung in
den meisten Fällen sehr einfach.  Der Quelltext in dieser Vorlage
lautet:
\begin{lstlisting}
\begin{pointlist}
  1ai:5, 1aii:3, 2:10,
  $\sum$ : +
\end{pointlist}
\end{lstlisting}

Die einzelnen (Teil-)Aufgaben sind durch Kommata getrennt.  Zuerst
kommt die Nummer der Aufgabe und anschließend nach einem Doppelpunkt
die Anzahl der erreichbaren Punkte.  Steht statt einer Punktzahl ein
"`\lstinline!+!"'-Zeichen, so werden alle Punktzahlen der vorherigen
Aufgaben aufsummiert.  Leerzeichen und einfache Zeilenumbrüche werden
ignoriert.

Eine Schwierigkeit könnte sich daraus ergeben, dass es bei einem
Übungszettel Bonus-Aufgaben gibt, welche die maximale Punktzahl nicht
erhöhen.  Dies lässt sich lösen, indem man diese Aufgaben bei der
Berechnung der Summe herausnimmt:
\begin{lstlisting}
\begin{pointlist}
  1:10, 2:10, 3:10, 4:10, 5:10, B:5!0,
  $\sum$ : +
\end{pointlist}
\end{lstlisting}
In diesem Beispiel bringen die Aufgaben 1 bis 5 jeweils 10 Punkte, und
die mit "`B"' bezeichnete Bonusaufgabe bringt 5 Punkte, wird aber für
die Summe mit 0 Punkten berechnet:
\begin{center}
\begin{pointlist}
  1:10, 2:10, 3:10, 4:10, 5:10, B:5!0,
  $\sum$ : +
\end{pointlist}
\end{center}

In vielen praktischen Fächern ist es der Fall, dass Theorie- und
Programmieraufgaben getrennt bewertet werden.  Nehmen wir an, ein
Übungszettel habe 3 Theorie- und 2 Programmieraufgaben:
\begin{lstlisting}
\begin{pointlist}
  1:10, 2:10, 3:10,
  $\sum$ : +,
  P1:10, P2:10,
  $\sum$ : +
\end{pointlist}
\end{lstlisting}
Durch die erste Summen-Ausgabe wird die Summe zurückgesetzt, sodass
die zweite Summe nur die Aufgaben P1 und P2 umfasst.
\begin{center}
\begin{pointlist}
  1:8, 2:8, 3:8,
  $\sum$ : +,
  P1:10, P2:10,
  $\sum$ : +
\end{pointlist}
\end{center}

Mithilfe eines optionalen Parameters lässt sich die Breite der Zellen
verändern.  Außerdem lassen sich Einträge durch Paare geschweifter
Klammern "`schützen"', sodass sie auch \lstinline|:!+,| enthalten
dürfen.  Letztendlich kann auch eine Summen-Spalte so angepasst
werden, dass sie einen anderen Wert enthält als dem berechneten:
\begin{center}
\begin{pointlist}[1cm]
  {1+}:9, {2,3}:7, $\sum$:0x10!+,
  {4:a}:5, 5:{3!}!6, $\sum$:+
\end{pointlist}
\end{center}
Der Quellcode hierzu lautet:
\begin{lstlisting}
\begin{pointlist}[1cm]
  {1+}:9, {2,3}:7, $\sum$:0x10!+,
  {4:a}:5, 5:{3!}!6, $\sum$:+
\end{pointlist}
\end{lstlisting}

\subsection{Fehler beim Kompilieren}

Sollten beim Kompilieren Fehler der Art
\begin{lstlisting}[numbers=none]
!pdfTeX error: pdflatex (file cm-super-ts1.enc): cannot open encoding file for reading
\end{lstlisting}
oder
\begin{lstlisting}[numbers=none]
!pdfTeX error: pdflatex (file sfbx1200.pfb): cannot open Type 1 font file for reading
\end{lstlisting}
auftreten, so ist die Schrift \lstinline[language=]!cm-super! nicht
installiert.  Mögliche Lösungen sind:

\begin{enumerate}
\item Installation von \lstinline[language=]!cm-super! (empfohlen).
  Die Schrift ist sowohl in \TeX\,Live als auch in MiK\TeX\ enthalten.

\item Falls stattdessen die Schrift \lstinline[language=]!lmodern!
  installiert ist, kann diese durch
\begin{lstlisting}[numbers=none]
\usepackage{lmodern}
\end{lstlisting}
aktiviert werden.  Allerdings sieht damit große Schrift verfälscht aus.

\item Zur Not kann auch auf die Standard-Kodierung zurückgegriffen
  werden.  Hierzu müssen in \lstinline[language=]!uebung42.tex! die
  Zeile
\begin{lstlisting}[numbers=none]
\usepackage[T1]{fontenc}
\end{lstlisting}
  sowie in \lstinline[language=]!soerenmath.tex! die Zeilen
\begin{lstlisting}[numbers=none]
\RequirePackage{textcomp}
\lstset{upquote=true}
\end{lstlisting}
  deaktiviert werden.
\end{enumerate}

\end{document}
