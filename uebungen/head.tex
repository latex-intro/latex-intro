\usepackage[utf8]{inputenc}
\usepackage[T1]{fontenc}
\usepackage[ngerman]{babel}
\usepackage{csquotes}
\usepackage{amsmath,amssymb,amsthm}
\usepackage{lmodern}
\usepackage{enumitem}
\usepackage{multirow}
\usepackage{url}
\usepackage[locale=DE,separate-uncertainty=true]{siunitx}
\usepackage[hidelinks]{hyperref}
\usepackage{textcomp}
\usepackage[version=3,arrows=pgf]{mhchem}
\usepackage{etoolbox}
\usepackage[framemethod=pgf]{mdframed}
\usepackage{environ}
\usepackage[european]{circuitikz}
\usepackage{tikz}
\usepackage{listings}
\usepackage[backend=biber,style=authoryear]{biblatex}

\addbibresource{files/bibliography.bib}

\usepackage[margin=1in]{geometry}

\usepackage[
	margin=0pt
	,oneside
	,textfont={small,it}
	,format=plain
	,indention=10pt
	,labelfont=sc
	,tableposition=top
	]{caption}

\newenvironment{uebung}{\begin{enumerate}[label=\thesection.\arabic*),ref=\thesection.\arabic*]}{\end{enumerate}}

\newenvironment{hinweis}{\begin{mdframed}[style=loesung,hidealllines=true]%
	\small
	\textbf{Hinweis:} }{\end{mdframed}}
	
\newenvironment{mathespecial}{}{}

\newenvironment{loesung}{\begin{mdframed}[style=loesung]\small\textbf{Lösung:} }{\end{mdframed}}
\NewEnviron{hide}{}
\newcommand{\hidesolutions}{
  \let\loesung\hide
  \let\endloesung\endhide
}

\newcommand{\hidespecialmath}{
  \let\mathespecial\hide
  \let\endmathespecial\endhide
}

\mdfdefinestyle{loesung}{leftmargin=1.5em,rightmargin=1.5em}

\newcommand\abs[1]{\lvert#1\rvert}
\newcommand\norm[1]{\lVert#1\rVert}
\newcommand\datei[1]{\textsf{#1}}

\providecommand{\course}{\!}
\AtBeginDocument{%
\begin{center}
	\LARGE\bfseries Übungen zum \LaTeX-Kurs \course{} \the\year\par
	\normalsize\normalfont
	\mbox{Arbeitsgruppe \LaTeX{}: R. Görmer, M. Gerken, Y. Schädler}
\end{center}
\begin{hinweis}
Viele der folgenden Aufgaben benötigen eigenständige Arbeit! Insbesondere musst
du selbstständig Anleitungen lesen. Dies ist keine Schikane, sondern soll
zeigen, dass \LaTeX\ sehr gut und ausführlich dokumentiert ist.
\end{hinweis}
}

\providecommand\textcolor{}
\renewcommand{\textcolor}[2]{#2}	% Farben deaktivieren

\newcommand{\cmd}[1]{\texttt{\textbackslash{#1}}}
\newcommand{\marg}[1]{\texttt{\{}\textlangle\emph{#1}\textrangle\texttt{\}}}
\newcommand{\oarg}[1]{\texttt{[}\textlangle\emph{#1}\textrangle\texttt{]}}
\newcommand{\pkg}[1]{\texttt{#1}}
\newcommand{\umg}[1]{\texttt{#1}}

\newcommand{\R}{\mathbb{R}}

\renewcommand{\i}{\mathfrak{i}}
\newcommand{\complex}[1]{\re\left(#1\right)+\i\im\left(#1\right)}
\DeclareMathOperator{\ArcSin}{ArcSin}
\DeclareMathOperator{\sgn}{sgn}

\DeclareMathOperator{\re}{Re}
\DeclareMathOperator{\im}{Im}

\addtokomafont{sectioning}{\rmfamily}
\addtokomafont{descriptionlabel}{\rmfamily}

%% Theorem Umgebungen: 
\newtheorem{mythm}{Satz}

\newtheoremstyle{others}  	% follow `plain` defaults but change HEADSPACE.
  {\baselineskip}   		% ABOVESPACE
  {\baselineskip}   		% BELOWSPACE
  {\upshape}  				% BODYFONT
  {}       					% INDENT (empty value is the same as 0pt)
  {\bfseries} 				% HEADFONT
  {:}         				% HEADPUNCT
  {.5em}  					% HEADSPACE. `plain` default: {5pt plus 1pt minus 1pt}
  {}          				% CUSTOM-HEAD-SPEC
  
\theoremstyle{others}
\newtheorem{mythm2}{Satz}[section]
