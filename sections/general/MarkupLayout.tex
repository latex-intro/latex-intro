\section{Markup und Layout}

\begin{frame}[fragile]
    \frametitle{Absätze und Umbrüche}
    
    Um einen neuen Absatz anzufangen, muss eine Leerzeile eingefügt werden
    
    \bigskip
    Einen normalen Umbruch erhält man mit \texttt{\textbackslash\textbackslash}
    
    \bigskip\pause
    Für eine neue Seite bzw. Spalte gibt es den Befehl \verb+\newpage+\\
    Alternativ: \verb+\clearpage+ beginnt immer eine neue Seite
    
    \bigskip\pause
    Um den Abstand zwischen den Absätzen zu vergrößern, kann an der Stelle einer der folgenden Befehle verwendet werden: \verb+\smallskip+, \verb+\medskip+, \verb+\bigskip+
\end{frame}


\begin{frame}
    \frametitle{Das Dokument gliedern}
    Um das Dokument zu teilen, können wir verschiedene Überschriften definieren:
    \medskip
    \begin{center}
        \begin{tabular}{ll}
            \textbf{Befehl} & \textbf{Verwendung}\\
            \cmd{part}\marg{Name} & Bücher\pause\\
            \cmd{chapter}\marg{Name} & Reports, Bücher\pause\\
            \cmd{section}\marg{Name} & Alle\pause\\
            \cmd{subsection}\marg{Name} & Alle\pause\\
            \cmd{subsubsection}\marg{Name} & Alle\pause\\
            \cmd{paragraph}\marg{Name} & Alle\pause\\
            \cmd{subparagraph}\marg{Name} & Alle
        \end{tabular}
    \end{center}
    \medskip\pause
    In Artikeln (\texttt{scrartcl}) werden meist nur die sections verwendet.
\end{frame}


\begin{frame}[fragile]
    \frametitle{Hervorhebungen im Text}
    
    Um Textstellen hervorzuheben, bietet \LaTeX{} einige Varianten:
    \medskip
    \begin{center}
        \begin{tabular}{ll}
            \textbf{Befehl} & \textbf{Ergebnis}\\
            \verb+\emph{Text}+ & \emph{Text}\pause\\
            \verb+\textbf{Text}+ & \textbf{Text}\pause\\
            \verb+\textit{Text}+ & \textit{Text}\pause\\
            \verb+\underline{Text}+ & \underline{Text}\pause\\
            \verb+\texttt{Text}+ & \texttt{Text}
        \end{tabular}
    \end{center}
    \pause\medskip
    
    Hervorhebungen sparsam verwenden, am besten nur \verb|\emph{}|.
\end{frame}


\begin{frame}[fragile]
    \frametitle{Inhaltsverzeichnis}
    Ein Inhaltsverzeichnis ist einfach mit \verb+\tableofcontents+ zu erhalten. Hierzu werden alle Überschriften verwendet.
    \medskip\pause
    
    Überschriften ohne Nummer und Eintrag im Inhalt erstellt man mit der gesternten Variante, z.B. \cmd{section}*\marg{Name}.
    
    \medskip\pause
    Genauso kann man Verzeichnisse von Tabellen und Abbildungen erzeugen. Dazu später mehr.
    
    \medskip\pause
    \begin{alertblock}{Achtung}
        Für Verzeichnisse benötigt \LaTeX{} einen zweiten Aufruf, nach dem ersten ist das Inhaltsverzeichnis noch leer.
    
        \smallskip
        Dies gilt auch für Änderungen an den Überschriften.
    \end{alertblock}
\end{frame}


\begin{frame}[fragile]
    \frametitle{Der Titel}
    
    \LaTeX{} kann den Titel automatisch setzen, dafür müssen aber die entsprechenden Angaben in der Präambel gemacht werden:
    \medskip
    \begin{center}
        \begin{tabular}{ll}
            \cmd{title}\marg{Name} & Setzt den Titel*\pause\\
            \cmd{author}\marg{Name} & Setzt den Autor*\pause\\
            \cmd{date}\marg{Datum} & Setzt das Datum\pause\\
            \cmd{subtitle}\marg{Name} & Setzt den Untertitel\pause\\
            \cmd{publishers}\marg{Name} & Setzt den Herausgeber
        \end{tabular}
    \end{center}
    \medskip
    Angaben mit einem Stern sind Pflicht.
    \medskip\pause
    
    
    Der Titel wird nun im Dokument mit \verb+\maketitle+ erstellt.
\end{frame}


\begin{frame}
    \frametitle{Umgebungen}
    \LaTeX{} steuert viele Dinge über Umgebungen.\\\pause
    Eine Umgebung wird mit
    \begin{center}
        \cmd{begin}\marg{Name}
    \end{center}
    gestartet\pause{} und mit
    \begin{center}
        \cmd{end}\marg{Name}
    \end{center}
    wieder beendet.
    
    \bigskip\pause
    Ihr kennt bereits die \umg{document}-Umgebung
\end{frame}


\begin{frame}[fragile]
    \frametitle{Textausrichtung}
    \framesubtitle{Zentriert mittels \umg{center}}

    \begin{center}
        \fbox{\begin{minipage}{0.7\linewidth}
            \begin{center}
                Text (und mehr) lässt sich mit der \umg{center}-Umgebung
                zentrieren, wenn gewünscht.
            \end{center}
        \end{minipage}}
    \end{center}
    \pause
    
    \begin{center}
    \begin{codeblock}
\begin{verbatim}
\begin{center}
    Text (und mehr) lässt sich
    mit der center-Umgebung
    zentrieren, wenn gewünscht.
\end{center}
\end{verbatim}
\end{codeblock}
    \end{center}
\end{frame}


\begin{frame}[fragile]
    \frametitle{Textausrichtung}
    \framesubtitle{Linksbündig mittels \umg{flushleft}}

    \begin{center}
        \fbox{\begin{minipage}{0.7\linewidth}
            \begin{flushleft}
                Standardmäßig benutzt \LaTeX{} Blocksatz. Wenn stattdessen ein linksbündiger Flattersatz verwendet werden soll, ist das natürlich auch möglich. Hierzu gibt es die \umg{flushleft}-Umgebung.
            \end{flushleft}
        \end{minipage}}
    \end{center}
    \pause
    
    \begin{center}
    \begin{codeblock}
\begin{verbatim}
\begin{flushleft}
    Standardmäßig benutzt \LaTeX{} Blocksatz.
    Wenn stattdessen ein linksbündiger
    Flattersatz verwendet werden soll...
\end{flushleft}
\end{verbatim}
    \end{codeblock}
    \end{center}
\end{frame}


\begin{frame}
    \frametitle{Aufzählungen und Listen}

    Aufzählungen sind ebenfalls Umgebungen.\\
    Dies bedeutet
    \begin{itemize}
        \item Man kann sie Schachteln
        \item Sie haben eine einfache Syntax
    \end{itemize}
    
    \bigskip\pause
    \LaTeX{} bietet die folgenden Umgebungen:
    \begin{itemize}
        \item \umg{itemize} erzeugt Bullets wie in dieser Liste\pause
        \item \umg{enumerate} erzeugt nummerierte Aufzählungen\pause
        \item \umg{description} erzeugt Erläuterungen
    \end{itemize}
\end{frame}


\begin{frame}[fragile]
    \frametitle{Aufzählungen und Listen}
    \framesubtitle{Ungeordnete Listen mit \umg{itemize}}
    
    Eine Aufzählung sieht so aus:
    \begin{itemize}
        \item Erster Eintrag
        \item Letzter Eintrag
    \end{itemize}
    
    \bigskip\pause
\begin{codeblock}
\begin{verbatim}
\begin{itemize}
    \item Erster Eintrag
    \item Letzter Eintrag
\end{itemize}
\end{verbatim}
\end{codeblock}
\end{frame}


\begin{frame}[fragile]
    \frametitle{Aufzählungen und Listen}
    \framesubtitle{Ungeordnete Listen mit \umg{itemize}}
    
    Schachtelung ist auch möglich
    \begin{itemize}
        \item Erster Eintrag
        \begin{itemize}
            \item Unterpunkt
            \item anderer Unterpunkt
        \end{itemize}
        \item Letzter Eintrag
    \end{itemize}
    
    \smallskip\pause
    \begin{verbatim}
Eine Aufzählung sieht so aus:
\begin{itemize}
    \item Erster Eintrag
    \begin{itemize}
        \item Unterpunkt
        \item anderer Unterpunkt
    \end{itemize}
    \item Letzter Eintrag
\end{itemize}
    \end{verbatim}
\end{frame}


\begin{frame}[fragile]
    \frametitle{Aufzählungen und Listen}
    \framesubtitle{Geordnete Listen mit \umg{enumerate}}
    
    Eine Aufzählung sieht so aus:
    \begin{enumerate}
        \item Erster Eintrag
        \item Letzter Eintrag
    \end{enumerate}
    
    \medskip\pause
\begin{codeblock}
    \begin{verbatim}
\begin{enumerate}
    \item Erster Eintrag
    \item Letzter Eintrag
\end{enumerate}
\end{verbatim}
\end{codeblock}
    \medskip
    
    Eine andere Umgebung, der Rest ist gleich
\end{frame}


\begin{frame}[fragile]
    \frametitle{Aufzählungen und Listen}
    \framesubtitle{Beschreibungen mit \umg{description}}
    
    Mit der \umg{description}-Umgebung kann man Erläuterungen erzeugen
    \medskip
    \begin{description}
        \item[\LaTeX{}] Ist ein tolles Programm zum Textsatz welches anfangs komplizierter ist, nachher aber viele Probleme vereinfacht
        \item[MS Word] Kann auch viel, nur nicht so schön
    \end{description}
    
    \bigskip\pause
\begin{codeblock}
\begin{verbatim}
\begin{description}
    \item[\LaTeX{}] Ist ein tolles Programm zum Textsatz
    \item[Microsoft Word] Kann auch viel, ...
\end{description}
\end{verbatim}
\end{codeblock}
\end{frame}


\begin{frame}[fragile]
    \frametitle{Zitate}
    \framesubtitle{Die \umg{quote}-Umgebung}
    
    Zitate können mit der \umg{quote}-Umgebung sehr einfach vom restlichen Text abgehoben werden:
    \begin{quote}
        Seht mich an, ein Gehirn von der Größe eines Planeten, und man verlangt von mir,
        euch in die Kommandozentrale zu bringen. Nennt man das vielleicht berufliche
        Erfüllung? Ich jedenfalls tu's nicht.
    \end{quote}
    
    \bigskip\pause
    \begin{codeblock}
\begin{verbatim}
\begin{quote}
    Seht mich an, ...
\end{quote}
\end{verbatim}
    \end{codeblock}
\end{frame}


\begin{frame}[fragile]
    \frametitle{Quelltexte mit \umg{verbatim}}
    
    Quelltext kann mit der \umg{verbatim}-Umgebung gesetzt werden:
    
    \medskip
    \begin{verbatim}
Mit \LaTeX{} kann man tolle Sachen machen.
    \end{verbatim}
    
    \pause
    \begin{codeblock}
\texttt{%
\textbackslash{}begin\{verbatim\}\\
Mit \textbackslash{}LaTeX\{\} kann man tolle Sachen machen.\\
\textbackslash{}end\{verbatim\}
}
    \end{codeblock}
    \pause
    \begin{alertblock}{Achtung}
        Es werden keine Zeilen mehr umgebrochen, alle Leerzeichen werden so abgedruckt wie im Quelltext, keine Befehle funktionieren mehr. Bessere Quelltextdarstellung z.B. mit dem Paket \pkg{listings}
    \end{alertblock}
\end{frame}


\begin{frame}[fragile]
    \frametitle{Verweise im Dokument}
    \begin{alertblock}{Problem}
        Man will auf ein Kapitel verweisen: \verb+siehe Kapitel 3+.\pause
        
        Nun fügt man ein neues Kapitel vor diesem ein und alle Verweise stimmen nicht mehr.
    \end{alertblock}
    
    \pause
    \begin{block}{Lösung: Labels und Verweise}
        \begin{itemize}
            \item Man kann an vielen Stellen mit \cmd{label}\marg{Bezeichnung} ein Label setzen\pause
            \item Man bekommt die Nummer des Kapitels, des Eintrags in einer \umg{enumerate}-Umgebung etc. nun mit \cmd{ref}\marg{Bezeichnung}
        \end{itemize}
    \end{block}
\end{frame}

\begin{frame}[fragile]
    \frametitle{Verweise im Dokument}
    \framesubtitle{Ein Beispiel}
    Ein Label kann zum Beispiel an eine Überschrift vergeben werden. Dann kann mit \cmd{ref} die Kapitelnummer und mit \cmd{pageref}\marg{Bezeichner} die Seitenzahl benutzt werden:
    \bigskip
    \pause

    \begin{verbatim}
\section{Test}\label{sec:Test}
In Kapitel \ref{sec:Test} auf Seite \pageref{sec:Test}
    \end{verbatim}
    
    \pause
    \begin{alertblock}{Achtung}
        So wie Verzeichnisse brauchen auch Verweise zwei Durchläufe von \LaTeX{}, sonst werden nur zwei Fragezeichen angezeigt!
    \end{alertblock}
\end{frame}


\begin{frame}[fragile]
    \frametitle{Tipps und Tricks}
    \framesubtitle{Worttrennungen}
    Manchmal sind die automatischen Worttrennungen oder Zeilenumbrüche unpassend.
    \medskip\pause
    
    Diese können mit einem geschützten Leerzeichen (\texttt{\~{}}) verhindert werden: \verb+Am 9.~September+ $\Rightarrow$ Am 9.~September
    
    \medskip\pause
    Außerdem ist es möglich, die Silbentrennung für einzelne Wörter anzupassen, wenn die automatische Trennung falsch ist.
    \smallskip
    
    Beispiel: \verb+Staats\-ver\-trag+ wird nur an den markierten Stellen getrennt.
\end{frame}

\begin{frame}[fragile]
    \frametitle{Tipps und Tricks}
    \framesubtitle{Fußnoten}
    Fußnoten kann man mit \cmd{footnote} einfügen.
    
    \bigskip
    Dann verwendet man einfach\\
    \verb|\footnote{Dies ist eine Fußnote}|
\end{frame}
