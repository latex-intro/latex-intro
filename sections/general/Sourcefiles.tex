\section{Syntax und Grundstruktur}

\begin{frame}[fragile]
    \frametitle{Zur Syntax}
    \begin{block}{Befehl}
        \cmd\Befehl\oarg{Optionen}\marg{Argument1}...\marg{ArgumentN}
    \end{block}
    \pause
    \begin{block}{Umgebung}
        \cmd\begin\marg{Umgebung}\oarg{Optionen} \ldots \cmd\end\marg{Umgebung}
    \end{block}
    \pause
    \begin{block}{Kommentar}
        \verb+% Dies ist ein Kommentar+
    \end{block}
\end{frame}


\begin{frame}[fragile]
    \frametitle{Aufbau der Quelldatei}
    \begin{itemize}
        \item Präambel
        \item \enquote{Inhalt}
    \end{itemize}
    \bigskip
    \pause
    
    \begin{block}{\texttt{Beispiel.tex}}
\begin{verbatim}
\documentclass{scrartcl}
% Ende der Präambel

\begin{document}
    Hallo Welt
\end{document}
\end{verbatim}
    \end{block}
\end{frame}


\begin{frame}[fragile]
    \frametitle{Was kommt wo hin?}
    Präambel (Header)
    \begin{itemize}
        \item Dokumentenklasse
        \item Benutzte Pakete
        \item Einstellungen für das gesamte Dokument
        \item Einige Metadaten, z.B. Autor, Titel, \ldots
    \end{itemize}
    \bigskip

    Der eigentliche Inhalt (Text, Abbildungen, \ldots) kommt zwischen \verb+\begin{document}+ und \verb+\end{document}+
\end{frame}


\begin{frame}[fragile]
    \frametitle{Die Dokumentenklasse}
    \begin{block}{Syntax}
        \cmd\documentclass\oarg{Optionen}\marg{Klasse}
    \end{block}
    \pause
    \begin{itemize}
        \item Dokumentklassen
        \begin{itemize}
            \item Die wichtigste Klasse: \verb+scrartcl+\pause
            \item Weitere: \verb+scrreprt+, \verb+scrbook+, \verb+scrlttr2+, \verb+beamer+\pause
        \end{itemize}
        \item Optionen:
        \begin{itemize}
            \item Schriftgröße: \verb+10pt+, \verb+11pt+, \verb+12pt+\pause
            \item Papiergröße: \verb+a4paper+\pause
            \item Zweispaltig: \verb+twocolumn+\pause
            \item Keine Einrückung am Anfang vom Absatz: \verb+parskip=full+
        \end{itemize}
    \end{itemize}
\end{frame}


\begin{frame}
    \frametitle{Die Dokumentenklassen im Überblick}

    \begin{tabular}{l|c|c}
        Dokumenttyp & KOMA-Klasse & Standard-Klasse\\
        \hline
        Einfacher Artikel & scrartcl & article\\
        längerer Report & scrreprt & report\\
        Buch & scrbook & book\\
        Briefe & scrlttr2 & --
    \end{tabular}
    
    \bigskip
    
    In der Regel sind die KOMA-Klassen den Standard-Klassen vorzuziehen, da die KOMA-Klassen an die deutsche Formatierung angepasst sind.
\end{frame}


\begin{frame}[fragile]
    \frametitle{Pakete einbinden}
    
    Latex kann mit Paketen erweitert werden. Dies ist für viele Dinge sehr wichtig, da die Standardfunktionen zum Teil nicht ausreichen.
    \begin{block}{Paket mit Optionen laden}
        \cmd\usepackage\oarg{Optionen}\marg{Paketname}
    \end{block}
    \pause
    \bigskip
    \begin{exampleblock}{Quelldatei ist UTF-8, immer nutzen}
		\verb+\usepackage[utf8]{inputenc}+
    \end{exampleblock}
    \pause
    \begin{exampleblock}{Sprache auf Deutsch umstellen}
        \verb+\usepackage[ngerman]{babel}+
    \end{exampleblock}
\end{frame}


\begin{frame}[fragile]
    \frametitle{Das babel-Paket}
    
    Sehr wichtiges Paket zur Lokalisierung:
    \begin{itemize}
        \item Deutsche Überschriften für z.B. Verzeichnisse
        \item Deutsche Silbentrennung
        \item Deutsche Anführungszeichen
    \end{itemize}
    \bigskip
    \begin{block}{Benutzen mit}
        \verb+\usepackage[ngerman]{babel}+
    \end{block}
\end{frame}


\begin{frame}[fragile]
    \frametitle{Umlaute}
    
    Umlaute funktionieren nicht Standardmäßig, Latex muss gesagt werden wie die Datei gespeichert ist:
    \begin{block}{Quelldatei ist UTF-8}
		\verb+\usepackage[utf8]{inputenc}+
    \end{block}
    \pause
    
    Eventuell muss Latex außerdem eine andere Schrift zur Ausgabe nutzen:
    \begin{block}{Andere Font-Darstellung}
		\verb+\usepackage[T1]{fontenc}+
    \end{block}
\end{frame}


\begin{frame}[fragile]
    \frametitle{Sonderzeichen}
    
    Latex nutzt die folgenden Sonderzeichen für spezielle Funktionen:
    \begin{center}
      \hfill \$
      \hfill \%
      \hfill \&
      \hfill \{
      \hfill \}
      \hfill \_
      \hfill \#
      \hfill \textbackslash
      \hfill \~{}
      \hfill \^{}
      \hfill
    \end{center}

    Um diese auszugeben, muss meist nur ein \textbackslash{} vor dieses gesetzt werden.
    Damit wird \verb+\$+ zu \$.
\end{frame}


\begin{frame}[fragile]
    \frametitle{Leerzeichen und Absätze}
    
    \begin{itemize}
        \item Mehrere Leerzeichen werden zu einem zusammengefasst \pause
        \item Ein neuer Absatz mit einer Leerzeile \pause
        \item Einen harten Umbruch erzeugt man mit \textbackslash\textbackslash
    \end{itemize}
    \pause    
    
    \begin{columns}[c]
        \column{.5\textwidth}
            \begin{verbatim*}
Dies ist ein Text  mit
vielen   Leerzeichen, die
verschwinden.
Und eine neue  Zeile.

Ein neuer Absatz geht auch!
\end{verbatim*}
        \pause
        \column{.5\textwidth}
        Dies ist ein Text  mit
        vielen   Leerzeichen, die
        verschwinden.
        Und eine neue  Zeile.

        Ein neuer Absatz geht auch!
\end{columns}
\end{frame}