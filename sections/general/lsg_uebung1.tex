\begin{loesung}
    \frametitle{Was passiert ohne \pkg{inputenc}}

    \begin{itemize}
        \item Ohne \pkg{inputenc} gehen die Umlaute nicht
        \item Mit \texttt{latin1} als Option gehen sie auch nicht \pause
        \item[\textrightarrow] Die Einstellung im Editor muss die selbe sein wie im Dokument
    \end{itemize}
    
    \bigskip
    Im Texmaker: Anzeige der Kodierung unten rechts in der Leiste.\\
    Man kann das unter \enquote{Optionen \textrightarrow{} Texmaker konfigurieren \textrightarrow{} Editor} ändern, es sollte immer auf UTF-8 stehen.
\end{loesung}

\begin{loesung}
    \frametitle{Wie setzt man \textbackslash{} und \^{}Hallo?}
    
    \begin{itemize}
        \item \textbackslash{} ist das einzige Zeichen, wo ein \textbackslash{}
            davor nicht zum gewünschten Ergebnis führt
        \item[\textrightarrow] Eigener Befehl: \cmd{textbackslash}\pause
        \medskip
        \item Das Dach funktioniert wie es soll, ergibt aber \^ Hallo
        \item[\textrightarrow] Damit das Zeichen davor bleibt, muss \cmd{\^{}\{\}}
            benutzt werden.\pause
        \medskip
        \item[\textrightarrow] Dies ist auch beim Backslash sinnvoll, da das
            nachfolgende Leerzeichen sonst verschwindet
    \end{itemize}
\end{loesung}