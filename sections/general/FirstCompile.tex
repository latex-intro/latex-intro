\section{Das erste Dokument}

\begin{frame}
    \frametitle{Los geht's!}
    \begin{itemize}
        \item Einen Ordner für diese Einführung anlegen
        \item Texmaker starten
        \item Den Texmaker auf UTF-8 umstellen
        \item Die Datei in diesem Ordner mit Namen \texttt{Beispiel.tex} abspeichern
    \end{itemize}
\end{frame}


\begin{frame}[fragile]
    \frametitle{Unser erstes Dokument}
    \begin{codeblock}[Beispiel.tex]
    \begin{verbatim}
\documentclass{scrartcl}

\begin{document}
    Hallo Welt
\end{document}
\end{verbatim}
    \end{codeblock}
\end{frame}


\begin{frame}[fragile]
    \frametitle{Das PDF erstellen}
    \begin{block}{Im Texmaker}
        \begin{itemize}
            \item Klick auf den Pfeil links neben \enquote{Schnelles Übersetzen}
            \item Die Taste F1 auf der Tastatur drücken
            \item Über das Menü \enquote{Werkzeuge}
        \end{itemize}
    \end{block}
    
    \bigskip
    \begin{block}{Alternativ im Terminal}
        \verb+pdflatex Beispiel.tex+
    \end{block}
\end{frame}


\begin{frame}
    \frametitle{Hilfe! Da ist alles Rot!}
    \begin{alertblock}{Wenn es doch nicht geht}
	    \begin{itemize}
	        \item Lest die Fehlermeldungen!
	        \item \texttt{undefined control sequence} heißt: Den Befehl gibt es nicht. Ihr habt ihn entweder falsch geschrieben oder ein benötigtes Paket vergessen
	        \item \texttt{missing \$ inserted} heißt: Ihr habt Befehle aus dem Mathemodus benutzt, ohne in diesen zu wechseln. Oder ihn vergessen zu beenden.
	    \end{itemize}
    \end{alertblock}
\end{frame}


\begin{frame}
    \frametitle{Hilfe zur Selbsthilfe}
    Wenn ihr den Fehler einfach nicht findet oder auch ein Problem nicht lösen könnt, versucht etwas von dem Folgenden:
    \begin{itemize}
        \item Mentor / Betreuer aus der Einführungswoche fragen
        \item Kommilitonen fragen
        \item Im Internet nach der Lösung suchen.
    \end{itemize}
    
    \bigskip
    Im Internet findet man außerdem sämtliche Handbücher.
\end{frame}