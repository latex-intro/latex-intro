\section{Mathematik-Modus}
\subsection{Umgebungen}
\begin{frame}[fragile]{Mathematische Grundlagen}
\framesubtitle{Das Paket \pkg{amsmath}}
\short{Wir wollen nun erläutern, wie man in \LaTeX{} Formeln schreibt.}{Wie schreibt man Formeln in \LaTeX{}?}

\medskip\pause
Wir laden das Paket \pkg{amsmath} durch \verb+\usepackage{amsmath}+. Mit diesem haben wir nun Zugriff auf eine Vielzahl mathematischer Befehle.

\medskip\pause
Formeln sehen anders aus, als der normale Textmodus (z.B. Bindestrich vs. Minus), deshalb müssen wir in \LaTeX{} erstmal den Mathe-Modus aktivieren.

\medskip\pause
Dafür gibt es verschiedene Möglichkeiten. Im Fließtext verwendet man typischerweise Dollarzeichen: \$ \emph{kurze Formel} \$

Einzeilige abgesetzte Formeln werden in eckigen Klammern geschrieben: 
\begin{center}
\textbackslash[\emph{Lange oder hervorgehobene Formel} \textbackslash]
\end{center}
\end{frame}

\begin{frame}[fragile]{Math. Grundlagen Beispiele}
\begin{columns}
\begin{column}{.48\textwidth}\footnotesize
\begin{codeblock}
\begin{verbatim}
	Text mit $a^2=\frac{1}{2}$
	Formel drin.

	Abgesetzte Formel:
	\[a^2=\frac{1}{2}\]
	Text nach der abgesetzten
	Formel.
\end{verbatim}
\end{codeblock}
\end{column}
%
\begin{column}{.45\textwidth}	\pause	
	Text mit $a^2=\frac{1}{2}$ Formel drin.
    
    \pause
	Abgesetzte Mathematik:
	\[a^2=\frac{1}{2}\]
	Text nach der abgesetzten Formel.
\end{column}
\end{columns}
\end{frame}





