\section{Mathematik-Modus}

\begin{frame}[fragile]{Mathematische Grundlagen}
\framesubtitle{Das Paket \textcolor{pkg}{amsmath}}
Wir wollen nun erläutern, wie man in \LaTeX{} Formeln schreibt. 

\medskip
Wir laden zunächst einmal das Paket \textcolor{pkg}{amsmath} mittels \pkg{amsmath}. Mit diesem haben wir nun Zugriff auf eine Vielzahl mathematischer Befehle. 

\medskip
Formeln sehen anders aus, als der normale Textmodus (z.B. Bindestrich vs. Minus), deshalb müssen wir in \LaTeX{} erstmal den Mathe-Modus aktivieren. 
\pause

\medskip
Dafür gibt es verschiedene Möglichkeiten. Im Fließtext verwendet man typischerweise Dollarzeichen: \$ \emph{kurze Formel} \$

Einzeilige abgesetzte Formeln werden in eckigen Klammern geschrieben: 
\begin{center}
\textbackslash[\emph{Lange oder hervorgehobene Formel} \textbackslash]
\end{center}
\end{frame}

\begin{frame}[fragile]{Math. Grundlagen Beispiele}

\begin{columns}
\begin{column}{.48\textwidth}\footnotesize
\begin{codeblock}
\begin{verbatim}
	Text mit $a^2=\frac{1}{2}$
	Formel drin.

	Abgesetzte Formel:
	\[a^2=\frac{1}{2}\]
	Text nach der abgesetzten
	Formel.
\end{verbatim}
\end{codeblock}
\end{column}
%
\begin{column}{.45\textwidth}		
	Text mit $a^2=\frac{1}{2})$ Formel drin.

	Abgesetzte Mathematik:
	\[a^2=\frac{1}{2}\]
	Text nach der abgesetzten Formel.
\end{column}

\end{columns}
\end{frame}

\begin{frame}[fragile]{Weitere Mathematik}
Häufig muss auf eine Formel Bezug genommen werden. \LaTeX{} hilft uns dabei mit Umgebungen, die Formeln durchnummerieren:
\begin{itemize}
\item \textcolor{umg}{equation}: Nummeriert einzeilig;
\item \textcolor{umg}{align}: Nummeriert mehrzeilig;
\item \textcolor{umg}{alignat}\cmd{}{Spaltenzahl}: Nummeriert mehrzeilig mit fester Spaltenzahl.
\item \textcolor{umg}{equation*}: Ohne Nummerierung einzeilig;
\item \textcolor{umg}{align*}: Ohne Nummerierung mehrzeilig;
\item \textcolor{umg}{alignat*}\cmd{}{Spaltenzahl}: Ohne Nummerierung mit fester Spaltenzahl.
\end{itemize}
\end{frame}

\begin{frame}[fragile]{Beispiele}
\begin{columns}
\begin{column}{.48\textwidth}\footnotesize
\begin{codeblock}
\begin{verbatim}
	Abgesetzt, nummeriert, 
	einzeilig:
    \begin{equation}
    a^2+b^2=c^2
    \end{equation}
    Abgesetzt, nummeriert,
     mehrzeilig:
    \begin{align}
    i^{2} = (-1)\\
    \exp{(i\pi)} = (-1) 
    \end{align}
\end{verbatim}
\end{codeblock}
\end{column}
%
\begin{column}{.45\textwidth}		
	Abgesetzt, nummeriert, einzeilig:
    \begin{equation}
    a^2+b^2=c^2
    \end{equation}
    Abgesetzt, nummeriert, mehrzeilig:
    \begin{align}
    i^{2} = (-1)\\
    \exp{(i\pi)} = (-1) 
    \end{align}
\end{column}

\end{columns}
\end{frame}
