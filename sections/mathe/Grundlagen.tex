\section{Mathematik-Modus}
\subsection{Umgebungen}
\begin{frame}[fragile]{Mathematische Grundlagen}
\framesubtitle{Das Paket \textcolor{pkg}{amsmath}}
Wir wollen nun erläutern, wie man in \LaTeX{} Formeln schreibt. 

\medskip\pause
Wir laden zunächst einmal das Paket \textcolor{pkg}{amsmath} mittels \upk{amsmath}. Mit diesem haben wir nun Zugriff auf eine Vielzahl mathematischer Befehle. 

\medskip\pause
Formeln sehen anders aus, als der normale Textmodus (z.B. Bindestrich vs. Minus), deshalb müssen wir in \LaTeX{} erstmal den Mathe-Modus aktivieren. 

\medskip\pause
Dafür gibt es verschiedene Möglichkeiten. Im Fließtext verwendet man typischerweise Dollarzeichen: \$ \emph{kurze Formel} \$

Einzeilige abgesetzte Formeln werden in eckigen Klammern geschrieben: 
\begin{center}
\textbackslash[\emph{Lange oder hervorgehobene Formel} \textbackslash]
\end{center}
\end{frame}

\begin{frame}[fragile]{Math. Grundlagen Beispiele}

\begin{columns}
\begin{column}{.48\textwidth}\footnotesize
\begin{codeblock}
\begin{verbatim}
	Text mit $a^2=\frac{1}{2}$
	Formel drin.

	Abgesetzte Formel:
	\[a^2=\frac{1}{2}\]
	Text nach der abgesetzten
	Formel.
\end{verbatim}
\end{codeblock}
\end{column}
%
\begin{column}{.45\textwidth}		
	Text mit $a^2=\frac{1}{2})$ Formel drin.

	Abgesetzte Mathematik:
	\[a^2=\frac{1}{2}\]
	Text nach der abgesetzten Formel.
\end{column}

\end{columns}
\end{frame}

\begin{frame}[fragile]{Weitere Mathematik}
Häufig muss auf eine Formel Bezug genommen werden. \LaTeX{} hilft uns dabei mit Umgebungen, die Formeln durchnummerieren:
\begin{itemize}
\item \textcolor{umg}{equation}: Nummeriert einzeilig;
\item \textcolor{umg}{align}: Nummeriert mehrzeilig;
\item \textcolor{umg}{alignat}\textrm{\{\textlangle Spaltenzahl\textrangle\}}: Nummeriert mehrzeilig mit fester Spaltenzahl.
\item \textcolor{umg}{equation*}: Ohne Nummerierung einzeilig;
\item \textcolor{umg}{align*}: Ohne Nummerierung mehrzeilig;
\item \textcolor{umg}{alignat*}\textrm{\{\textlangle Spaltenzahl\textrangle\}}: Ohne Nummerierung mit fester Spaltenzahl.
\end{itemize}
\end{frame}

\begin{frame}[fragile]{Beispiele}
\begin{columns}
\begin{column}{.48\textwidth}\footnotesize
\begin{codeblock}
\begin{verbatim}
	Abgesetzt, nummeriert, 
	einzeilig:
    \begin{equation}
    a^2+b^2=c^2
    \end{equation}
    Abgesetzt, nummeriert,
     mehrzeilig:
    \begin{align}
    i^{2} = (-1)\\
    \exp{(i\pi)} = (-1) 
    \end{align}
\end{verbatim}
\end{codeblock}
\end{column}
%
\begin{column}{.45\textwidth}		
	Abgesetzt, nummeriert, einzeilig:
    \begin{equation}
    a^2+b^2=c^2
    \end{equation}
    Abgesetzt, nummeriert, mehrzeilig:
    \begin{align}
    i^{2} = (-1)\\
    \exp{(i\pi)} = (-1) 
    \end{align}
\end{column}
\end{columns}
\end{frame}

\begin{frame}[fragile]{Die \textcolor{umg}{alignat}-Umgebung}
\framesubtitle{Aussehen} 
Die \textcolor{umg}{alignat}-Umgebung empfiehlt sich vor allem bei Umformen von Gleichungen. Ein Beispiel:
\begin{columns}
\begin{column}{.5\textwidth}
\begin{codeblock}
\begin{verbatim}
\begin{alignat}{2}
&& a &= b + c\\
\iff \quad && a - c &= b\\
\iff \quad && c &= a - b
\end{alignat}
\end{verbatim}
\end{codeblock}
\end{column}

\begin{column}{.4\textwidth}
Wir erhalten:
\begin{alignat}{2}
&& a &= b + c\\
\iff \quad && a - c &= b\\
\iff \quad && c &= a - b
\end{alignat}
\end{column}
\end{columns}
\pause
Beachte: \alert{Am ende der letzten Zeile darf kein \textbackslash\textbackslash{} stehen!}
\end{frame}

\begin{frame}[fragile]{Die \umg{alignat}-Umgebung}
\framesubtitle{Aussehen}
Betrachten wir nun folgendes Beispiel: 

\begin{codeblock}
\begin{verbatim}
\begin{alignat}{4}
a + (3 + 1) &= a + 4          &=&a + (2 + 2)\\
            &= a + (0 + 4)    &=&a + (-1 + 5)\\
b + 102     &= b + 100        &=&b + (49 + 51)\\
            &= b + 1000 - 900 &=&b + (49 + 51)
\end{alignat}
\end{verbatim}
\end{codeblock}
\end{frame}

\begin{frame}{Die \umg{alignat}-Umgebung}
\framesubtitle{Aussehen}
\begin{alignat}{4}
a + (3 + 1) &= a + 4          &=&a + (2 + 2)\\
            &= a + (0 + 4)    &=&a + (-1 + 5)\\
b + 102     &= b + 100        &=&b + (49 + 51)\\
            &= b + 1000 - 900 &=&b + (49 + 51)
\end{alignat}
Was fällt auf?

\medskip
\pause
Die \umg{alignat}-Umgebung wechselt stets zwischen rechts-links-Ausrichtung.
\end{frame}
