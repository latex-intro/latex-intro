\section{Mathematik-Modus}
\subsection{Umgebungen}
\begin{frame}[fragile]{Mathematische Grundlagen}
\framesubtitle{Das Paket \pkg{amsmath}}
Wie schreibt man Formeln in \LaTeX{}?

\medskip\pause
Zugriff auf viele mathematische Befehle durch das \pkg{amsmath}-Paket

\medskip\pause
Formeln sehen anders aus, deshalb müssen wir in \LaTeX{} den Mathe-Modus aktivieren.

\medskip\pause
Verschiedene Möglichkeiten: Im Fließtext Dollarzeichen: \texttt{\$} \emph{kurze Formel} \texttt{\$}

Einzeilige abgesetzte Formeln in eckigen Klammern: 
\begin{center}
\texttt{\textbackslash[} \emph{Lange oder hervorgehobene Formel} \texttt{\textbackslash]}
\end{center}
\end{frame}

\begin{frame}[fragile]{Math. Grundlagen Beispiele}
\begin{columns}
\begin{column}{.48\textwidth}\footnotesize
\begin{codeblock}
\begin{verbatim}
	Text mit $a^2=\frac{1}{2}$
	Formel drin.

	Abgesetzte Formel:
	\[a^2=\frac{1}{2}\]
	Text nach der abgesetzten
	Formel.
\end{verbatim}
\end{codeblock}
\end{column}
%
\begin{column}{.45\textwidth}	\pause	
	Text mit $a^2=\frac{1}{2}$ Formel drin.
    
    \pause
	Abgesetzte Mathematik:
	\[a^2=\frac{1}{2}\]
	Text nach der abgesetzten Formel.
\end{column}
\end{columns}
\end{frame}





