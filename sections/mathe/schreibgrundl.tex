\subsection{Wie nutzt man Mathe}

\begin{frame}[fragile]{Wie mache ich Mathe?}
Einfache Beispiele für Mathe-Modus
\begin{columns}
\begin{column}{0.55\textwidth}
\begin{codeblock}
\begin{verbatim}
Hochgestellt $a^{hoch}$ 
 vs. $a^bc$
Index $a_{Index}$ 
 vs. $a_bc$
Summe $\sum$, Integral $\int$
Grenzen $\sum_a^b$ 
 vs. $\displaystyle\sum_a^b$
Gr. Buchstaben $\alpha, \beta$
\end{verbatim}
\end{codeblock}
\end{column}
\pause
\begin{column}{0.4\textwidth}
Hochgestellt $a^{hoch}$ vs. $a^bc$. 

\medskip\pause
Index $a_{Index}$ vs. $a_bc$.

\medskip\pause
Summe $\sum$, Integral $\int$.

\medskip\pause
Grenzen $\sum_a^b$ vs. $\displaystyle \sum_a^b$

\medskip\pause
Gr. Buchstaben $\alpha, \beta$.
\end{column}
\end{columns}
\end{frame}


\begin{frame}[fragile]{Wie mache ich Mathe II}
\begin{columns}
\begin{column}{0.55\textwidth}
\begin{codeblock}
\begin{verbatim}
Physikalische Ableitungen 
 $\dot{x}$ und $\ddot{x}$. 
Vektoren $\vec{a}$.
Ableitungsoperator $\partial$.
\end{verbatim}
\end{codeblock}
\end{column}
\pause
\begin{column}{0.4\textwidth}
Physikalische Ableitungen $\dot{x}$ und $\ddot{x}$. 

\medskip\pause
Vektoren $\vec{a}$.

\medskip\pause
Ableitungsoperator $\partial$.
\end{column}
\end{columns}
\end{frame}

\begin{frame}[fragile]{Brüche}
Brüche über den Befehl 
\begin{center}
\cmd{frac}\marg{Zähler}\marg{Nenner}
\end{center}
einen Bruch im Mathe-Modus:
\begin{columns}
\begin{column}{0.55\textwidth}
\begin{codeblock}
\begin{verbatim}
\frac{\exp(i\pi)}{i^2} = 1
\end{verbatim}
\end{codeblock}
\end{column}
\begin{column}{0.35\textwidth}
Das Ergebnis sieht dann so aus:
\[
    \frac{\exp(i\pi)}{i^2} = 1
\]
\end{column}
\end{columns}
\end{frame}


\begin{frame}[fragile]{Abstände in \LaTeX}
\LaTeX{} ignoriert Leerzeichen im Mathemodus.

\medskip\pause 
Manuelle Leerzeichen:
\begin{center}
\begin{tabular}{p{0.2\textwidth}p{0.6\textwidth}}
Befehl & Bedeutung\\
\verb+\,+ & Setzt ein Leerzeichen. \\
\verb+\quad+ & Setzt einen Abstand von 1em. \\
\verb+\qquad+ & Setzt einen Abstand von 2em.
\end{tabular}
\end{center}\pause
\begin{columns}
\begin{column}{0.4\textwidth}
\begin{codeblock}
\begin{verbatim}
a\,b\quad c \qquad d
\end{verbatim}
\end{codeblock}
\end{column}
\begin{column}{0.35\textwidth}
Beachte größer werdende Abstände: 
\[
a\,b\quad c \qquad d
\]
\end{column}
\end{columns}
\end{frame}



