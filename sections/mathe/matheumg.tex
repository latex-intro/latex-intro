\begin{frame}[fragile]{Weitere Mathematik}
Häufig muss auf eine Formel Bezug genommen werden. \LaTeX{} hilft uns dabei mit Umgebungen, die Formeln durchnummerieren:

\pause
\begin{itemize}
\item \umg{equation}: Nummeriert einzeilig;\pause
\item \umg{align}: Nummeriert mehrzeilig;\pause
\item \umg{alignat}\textrm{\{\textlangle Spaltenzahl\textrangle\}}: Nummeriert mehrzeilig mit festerSpaltenzahl.\pause
\item \umg{equation*}: Ohne Nummerierung einzeilig;
\item \umg{align*}: Ohne Nummerierung mehrzeilig;
\item \umg{alignat*}\textrm{\{\textlangle Spaltenzahl\textrangle\}}: Ohne Nummerierung mit fester Spaltenzahl.
\end{itemize}
\end{frame}

\begin{frame}[fragile]{Beispiele}
\begin{columns}
\begin{column}{.48\textwidth}\footnotesize
\begin{codeblock}
\begin{verbatim}
	Abgesetzt, nummeriert, 
	einzeilig:
    \begin{equation}
    a^2+b^2=c^2
    \end{equation}
    Abgesetzt, nummeriert,
     mehrzeilig:
    \begin{align}
    i^{2} = (-1)\\
    \exp{(i\pi)} = (-1) 
    \end{align}
\end{verbatim}
\end{codeblock}
\end{column}
%
\begin{column}{.45\textwidth}	\pause	
	Abgesetzt, nummeriert, einzeilig:
    \begin{equation}
    a^2+b^2=c^2
    \end{equation}\pause
    Abgesetzt, nummeriert, mehrzeilig:
    \begin{align}
    i^{2} = (-1)\\
    \exp{(i\pi)} = (-1) 
    \end{align}
\end{column}
\end{columns}
\end{frame}