\subsection{Wie nutzt man Mathe}

\begin{frame}[fragile]{Wie mache ich Mathe?}
Wir wissen nun, wo wir Mathematik hinschreiben, aber noch nicht wirklich wie.
\begin{columns}
\begin{column}{0.55\textwidth}
\begin{codeblock}
\begin{verbatim}
Hochgestellt $a^{hoch}$ 
 vs. $a^bc$
Index $a_{Index}$ 
 vs. $a_bc$
Summe $\sum$, Integral $\int$
Grenzen $\sum_a^b$ 
 vs. $\displaystyle\sum_a^b$
Gr. Buchstaben $\alpha, \beta$
\end{verbatim}
\end{codeblock}
\end{column}
\pause
\begin{column}{0.4\textwidth}
Hochgestellt $a^{hoch}$ vs. $a^bc$. 

\medskip\pause
Index $a_{Index}$ vs. $a_bc$.

\medskip\pause
Summe $\sum$, Integral $\int$.

\medskip\pause
Grenzen $\sum_a^b$ vs. $\displaystyle \sum_a^b$

\medskip\pause
Gr. Buchstaben $\alpha, \beta$.
\end{column}
\end{columns}
\end{frame}

\begin{frame}[fragile]{Brüche}
Für Brüche benötigt man in \LaTeX{} weder Sonderzeichen, noch diverse andere Methoden. Brüche heißen im Englischen \glqq fraction\grqq{} und auf naive Weise erhält man durch den Befehl
\begin{center}
\textrm{\textbackslash frac\{\textlangle Zähler\textrangle\}\{\textlangle Nenner \textrangle\}}
\end{center}
einen Bruch im Mathe-Modus:
\begin{columns}
\begin{column}{0.55\textwidth}
\begin{codeblock}
\begin{verbatim}
\frac{\exp(i\pi)}{i^2} = 1
\end{verbatim}
\end{codeblock}
\end{column}
\begin{column}{0.35\textwidth}
Das Ergebnis sieht dann so aus:
\[
    \frac{\exp(i\pi)}{i^2} = 1
\]
\end{column}
\end{columns}
\end{frame}

\begin{frame}[fragile]{Schriftarten}
Im Mathe-Modus gibt es verschiedene Schriftarten. Zum Beispiel möchte man die Notation für bekannte Mengen, wie $\mathbb{R}$ oder $\mathbb{N}$, auch beibehalten. Die beiden wichtigsten Schriftarten im Mathe-Modus sind \cmd{mathbb}{Text} aus dem \pkg{amssymb}-Paket und \cmd{mathcal}{Text}.
\begin{columns}
\begin{column}{0.55\textwidth}
\begin{codeblock}
\begin{verbatim}
$\mathcal{NZQRC}$

$\mathbb{NZQRC}$
\end{verbatim}
\end{codeblock}
\end{column}
\begin{column}{0.35\textwidth}
$\mathcal{NZQRC}$

\medskip
$\mathbb{NZQRC}$
\end{column}
\end{columns}
\end{frame}

\begin{frame}[fragile]{Mengen in \LaTeX{}}
Frage: \LaTeX{} erwartet Parameter in geschweiften Klammern. Wie schreibe ich dann Mengenklammern? 

\medskip\pause
Antwort: Man stellt den geschweiften Klammern ein \textbackslash{} voran. 

\medskip\pause
Beispiel: \textrm{\$\textbackslash\{1, 2, 3\textbackslash\}\$} erzeugt dann $\{1, 2, 3\}$

\medskip\pause
Frage: Wie schreibe ich $x\in\mathbb{R}$

\medskip\pause
Antwort: Das $\in$-Symbol erhält man über die Befehle \textrm{\textbackslash in}($\in$) und \textrm{\textbackslash ni}($\ni$)

\vskip1em\pause
Bemerke: Möchte man Vorprogrammierte Sonderzeichen, wie z.B. \$, ausgeben, dann stellt man diesen ein \textbackslash{} voran.
\end{frame}

\begin{frame}[fragile]{Klammersetzung}
\LaTeX{} verändert die Klammergröße standardmäßig \alert{nicht}! Zum Beispiel: 
\begin{columns}
\begin{column}{0.55\textwidth}
\begin{codeblock}
\begin{verbatim}
\{(x,y)\in\mathbb{R}^2|
(\frac{x}{y})^2 = 1\}
\end{verbatim}
\end{codeblock}
\end{column}
\begin{column}{0.35\textwidth}
\[
    \{(x,y)\in\mathbb{R}^2|(\frac{x}{y})^2 = 1\}
\]
\end{column}
\end{columns}

\medskip\pause
Wir können allerdings über die Befehle \cmdd{left} und \cmdd{right} die Größe der Klammern automatisch einstellen.
\end{frame}

\begin{frame}[fragile]{Klammersetzung}
\framesubtitle{Beispiel}
\begin{columns}
\begin{column}{0.55\textwidth}
\begin{codeblock}
\begin{verbatim}
\left\{(x,y)\in\mathbb{R}^2
\middle|\left(\frac{x}{y}
\right)^2 = 1\right\}
\end{verbatim}
\end{codeblock}
\end{column}
\begin{column}{0.35\textwidth}
\[
    \left\{(x,y)\in\mathbb{R}^2\middle|\left(\frac{x}{y}\right)^2 = 1\right\}
\]
\end{column}
\end{columns}\pause
Wir sehen, dass die Befehle \cmdd{left} und \cmdd{right} die Größe der Klammern relativ zu dem Inhalt einstellt und mithilfe von \cmdd{middle} erhält man ein entsprechend großes Zeichen innerhalb der Klammern. 

\medskip\pause
Beachte: \alert{Ein \cmdd{left}-Befehl benötigt immer einen \cmdd{right}-Befehl!}
\end{frame}

\begin{frame}[fragile]{Klammersetzung II}
\framesubtitle{Feste Klammergröße}
Betrachten wir einmal folgendes Beispiel:
\[
    (x(y(3z(2x+y)-3)7z)4-8y)
\]

\medskip\pause
Problem: Alle Klammern sind Gleichgroß und der Ausdruck wird unleserlich. Die Befehle \cmdd{left} und \cmdd{right} Helfen hier nicht!

\medskip\pause
Lösung: Die Befehle \cmdd{big}, \cmdd{Big}, \cmdd{bigg} und \cmdd{Bigg}. Diese Befehle vergrößern die Klammern mit jedem weiteren Schritt
\end{frame}

\begin{frame}[fragile]{Klammersetzung II}
\framesubtitle{Beispiel}
\begin{codeblock}
\begin{verbatim}
\Biggl(x\biggl(y\Bigl(3z\bigl(
2x+y\bigr)-3\Bigr)7z\biggr)4-8y\Biggr)
\end{verbatim}
\end{codeblock}
Erzeugt Folgende Ausgabe: 
\[
    \Biggl(x\biggl(y\Bigl(3z\bigl(2x+y\bigr)-3\Bigr)7z\biggr)4-8y\Biggr)
\]\pause

Mit einer entsprechende Verknüpfung von \cmdd{left},\cmdd{right} und \cmdd{big}, etc., könnt ihr eure Formeln also übersichtlich halten. 
\end{frame}

\begin{frame}[fragile]{Matrizen}
Matrizen könnt ihr in \LaTeX{} ähnlich wie Tabellen setzen. In der Regel sollten Matrizen im abgesetzten Modus geschrieben werden. 
\begin{columns}
\begin{column}{0.55\textwidth}
\begin{codeblock}
\begin{verbatim}
\begin{pmatrix}
a_{11} & a_{12} & \cdots 
& a_{1n} \\
0 & a_{22} & & \vdots \\
\vdots & \ddots & \ddots 
& \vdots \\
0 & \cdots & 0 & a_{nn} \\
\end{pmatrix}
\end{verbatim}
\end{codeblock}
\end{column}
\begin{column}{0.35\textwidth}
\[
\begin{pmatrix}
a_{11} & a_{12} & \cdots & a_{1n} \\
0 & a_{22} & & \vdots \\
\vdots & \ddots & \ddots & \vdots \\
0 & \cdots & 0 & a_{nn} \\
\end{pmatrix}
\]
\end{column}
\end{columns}\pause
\end{frame}

\begin{frame}[fragile]{Buchstaben in Formeln}
Es gibt mehrere Zeichen, die in Formeln verwendet werden können. 

Zahlen: 
\begin{center}
\begin{tabular}{p{0.45\textwidth}p{0.4\textwidth}}
Beachte fehlende Leerzeichen: \textrm{\$1 2 34 \cmdd{infty}\$} & Beachte fehlende Leerzeichen: $1 2 34 \infty$
\end{tabular}
\end{center}
Buchstaben:
\begin{center}
\begin{tabular}{p{0.45\textwidth}p{0.4\textwidth}}
Text in Formeln:   & Text in Formeln:  \\
\textrm{\$Ergebnis\cmdd{neq} \cmdd{text}\{Ergebnis\}\$} & $Ergebnis\neq\text{Ergebnis}$
\end{tabular}
\end{center}
\end{frame}

\begin{frame}[fragile]{Griechische Buchstaben}
In \LaTeX{} kann man natürlich auch griechische Buchstaben verwenden: 
\begin{center}
\begin{tabular}{p{0.45\textwidth}p{0.4\textwidth}}
Griechen in groß:  & Griechen in groß: \\
\textrm{\$\cmdd{Phi},\cmdd{Lamda},\cmdd{Omega}\$} & $\Phi,\,\Lambda,\,\Omega$\\
\text{ }    &  \text{ }  \\
Griechen in klein:  & Griechen in klein: \\
\textrm{\$\cmdd{phi},\cmdd{lambda},\cmdd{omega} aber: \cmdd{varphi} und \cmdd{varepsilon}\$} & $\phi,\lambda,\omega$ aber: $\varphi$ und $\varepsilon$. 
\end{tabular}
\end{center}
\end{frame}
