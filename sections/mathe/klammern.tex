\begin{frame}[fragile]{Klammersetzung}
\LaTeX{} verändert die Klammergröße standardmäßig \alert{nicht}! Zum Beispiel: 
\begin{columns}
\begin{column}{0.55\textwidth}
\begin{codeblock}
\begin{verbatim}
\{(x,y)\in\mathbb{R}^2|
(\frac{x}{y})^2 = 1\}
\end{verbatim}
\end{codeblock}
\end{column}
\begin{column}{0.35\textwidth}
\[
    \{(x,y)\in\mathbb{R}^2|(\frac{x}{y})^2 = 1\}
\]
\end{column}
\end{columns}

\medskip\pause
Lösung: Die Befehle \cmd{left}, \cmd{middle]} und \cmd{right}.
\end{frame}

\begin{frame}[fragile]{Klammersetzung}
\framesubtitle{Beispiel}
\begin{columns}
\begin{column}{0.55\textwidth}
\begin{codeblock}
\begin{verbatim}
\left\{(x,y)\in\mathbb{R}^2
\middle|\left(\frac{x}{y}
\right)^2 = 1\right\}
\end{verbatim}
\end{codeblock}
\end{column}
\begin{column}{0.35\textwidth}
\[
    \left\{(x,y)\in\mathbb{R}^2\middle|\left(\frac{x}{y}\right)^2 = 1\right\}
\]
\end{column}
\end{columns}\pause
Wir sehen, dass die Befehle \cmd{left} und \cmd{right} die Größe der Klammern relativ zu dem Inhalt einstellt und mithilfe von \cmd{middle} erhält man ein entsprechend großes Zeichen innerhalb der Klammern.

\medskip\pause
Beachte: \alert{Ein \cmd{left}-Befehl benötigt immer einen \cmd{right}-Befehl!}
\end{frame}

\begin{frame}[fragile]{Klammersetzung II}
\framesubtitle{Feste Klammergröße}
Betrachten wir einmal folgendes Beispiel:
\[
    (x(y(3z(2x+y)-3)7z)4-8y)
\]

\medskip\pause
Problem: Alle Klammern sind Gleichgroß und der Ausdruck wird unleserlich. Die Befehle \cmd{left} und \cmd{right} Helfen hier nicht!

\medskip\pause
Lösung: Die Befehle \cmd{big}, \cmd{Big}, \cmd{bigg} und \cmd{Bigg}. Diese Befehle besitzen eine feste Größe in den Klammern.
\end{frame}

\begin{frame}[fragile]{Klammersetzung II}
\framesubtitle{Beispiel}
\begin{codeblock}
\begin{verbatim}
\Biggl(x\biggl(y\Bigl(3z\bigl(
2x+y\bigr)-3\Bigr)7z\biggr)4-8y\Biggr)
\end{verbatim}
\end{codeblock}
Erzeugt Folgende Ausgabe: 
\[
    \Biggl(x\biggl(y\Bigl(3z\bigl(2x+y\bigr)-3\Bigr)7z\biggr)4-8y\Biggr)
\]\pause

Übersichtliche Formeln durch Verknüppfung von \cmd{left}, \cmd{right} etc.
\end{frame}