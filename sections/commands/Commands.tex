\section{Eigene Befehle}

\begin{frame}[fragile]{Befehle definieren}
Wenn ihr viel mit \LaTeX{} arbeitet, werdet ihr zwangsläufig auf Ausdrücke stoßen, die wiederholt eingegeben werden. 

\medskip\pause
Das dauert oft lange und wir können uns die Zeit sparen, indem wir über den Befehl 
\begin{center}
\cmd{newcommand}\oarg{Anzahl Parameter}\marg{Befehlsname}\marg{Definition}
\end{center}
neue Befehle definieren. Dieser Befehl gehört in die Präambel! 
\begin{columns}
\begin{column}{0.55\textwidth}
\begin{codeblock}
\begin{small}
\begin{verbatim}
% Präambel 
\newcommand{\R}{\mathbb{R}}
...
\begin{document}
Das ist jetzt sehr einfach: $\R$.
\end{document}
\end{verbatim}
\end{small}
\end{codeblock}
\end{column}
\begin{column}{0.35\textwidth}
Das ist jetzt sehr einfach: $\R$.
\end{column}
\end{columns}
\end{frame}

\begin{frame}[fragile]{Wichtig für eigene Befehle}
\LaTeX{} warnt euch davor, wenn ihr einen vorhandenen Befehl überschreiben wollt. Ihr erhaltet die Fehlermeldung
\begin{center}
\textcolor{red}{\LaTeX{} Error: Command ... already defined.}
\end{center}\pause
Seid ihr euch sicher, dass der Befehel, den ihr überschreiben wollt, von euch nicht genutzt wird, dann könnt ihr diesen mit 
\begin{center}
\cmd{renewcommand}\oarg{Anzahl Parameter}\marg{Befehlsname}\marg{Definition}
\end{center}
überschreiben. Ihr solltet euch dabei aber absolut sicher sein, dass ihr den überschriebenen Befehl später nicht mehr benötigt! 
\end{frame}