\section{Grafiken}

\subsection{Grafiken einbinden}
\begin{frame}[fragile]
\frametitle{Grafiken einbinden}
\framesubtitle{Das Paket \pkg{graphicx}}
Mit dem Befehl \cmd{includegraphics} können Grafiken eingebunden werden.
\bigskip\pause

\textbf{Unterstützte Formate:}\smallskip
  \begin{itemize}
    \item[PNG] Portable Network Graphics
    \begin{itemize}
      \item verlustfreie Kompression
      \item Raster-/Pixelgrafik
    \end{itemize}\pause
    \item[JPG] Joint Photographic Experts Group
    \begin{itemize}
      \item verlustbehaftete Kompression
      \item Raster-/Pixelgrafik
    \end{itemize}\pause
    \item[PDF] Portable Document Format
    \begin{itemize}
      \item verlustfreie Kompression
      \item vektorbasiert, daher meist sehr gut skalierbar
    \end{itemize}
  \end{itemize}
\end{frame}

\begin{frame}[fragile]
\frametitle{Grafiken einbinden}
\framesubtitle{Die \umg{figure}-Umgebung}
\begin{itemize}
    \item Damit Grafiken ideal positioniert werden können, gibt es spezielle
        Gleit-Umgebungen (engl. \enquote{floats})
    \item \LaTeX{} platziert diese automatisch an der nächsten passenden Stelle
    \item Die Umgebung kann auch Bildunterschriften enthalten
    \item Die Positionierung kann in Grenzen beeinflusst werden
\end{itemize}
\end{frame}

\subsection{includegraphics}
\begin{frame}[fragile]
\frametitle{Grafiken einbinden}
\begin{codeblock}
\begin{verbatim}
\begin{figure}
\includegraphics[width=0.5\textwidth]{xkcd.png}
\end{figure}
\end{verbatim}
\end{codeblock}
\pause 
 \begin{figure}
      \includegraphics[width=0.5\textwidth]{images/xkcd.png}
 \end{figure}
\end{frame}

\subsection{Bildunterschrift}
\begin{frame}[fragile]
\frametitle{Bildunterschrift}
 
\begin{codeblock}
\begin{verbatim}
\begin{figure}
\includegraphics[width=0.5\textwidth]{xkcd.png}
\caption{Beispielbild von XKCD.com}
\end{figure}
\end{verbatim}
\end{codeblock}
    \begin{figure}
      \includegraphics[width=0.3\textwidth]{xkcd.png}
      \caption{Beispielbild von XKCD.com}
    \end{figure}
\end{frame}

\subsection{Positionierung von Abbildungen}
\begin{frame}[fragile]
\frametitle{Positionierung von Abbildungen}
  \begin{codeblock}
\begin{verbatim}
\begin{figure}[htbp]
\includegraphics[width=0.5\textwidth]{xkcd.png}
\end{figure}
\end{verbatim}
  \end{codeblock}
  \pause
  
  \begin{itemize}
    \item[h] (here) Positioniert bevorzugt an der Textstelle, an
        der die Umgebung steht\pause
    \item[t] (top) Positioniert bevorzugt am Seitenanfang\pause
    \item[b] (bottom) Positioniert bevorzugt am Seitenende\pause
    \item[p] (page) Positioniert auf neuer Seite
  \end{itemize}
\end{frame}