\section{Tabellen}

\subsection{Grundlage}
\begin{frame}[fragile]
\frametitle{Tabellen}
\framesubtitle{Die \umg{tabular}-Umgebung}
\begin{codeblock}
\verb|\begin{tabular}|\marg{Spaltendefinition}\\
\verb|  |\textlangle\emph{Tabelleninhalt}\textrangle\\
\verb|\end{tabular}|
\end{codeblock}
\pause\bigskip
  
\begin{itemize}
    \item \textit{Spaltendefinition}: \texttt{r}, \texttt{l}, \texttt{c},
        jeder Buchstabe eine Spalte\pause
    \item Senkrechte Linien mit \verb+|+\pause
    \item \textit{Tabelleninhalt}: Spalten werden mit \verb|&| getrennt,
        Zeilen mit \verb|\\|\pause
    \item Horizontale Linien mit \cmd{hline}
\end{itemize}
\end{frame}

\begin{frame}[fragile]
\frametitle{Tabellen}
\framesubtitle{Ein Beispiel}
\begin{codeblock}
\begin{verbatim}
\begin{tabular}{|l|c|r|}
 \hline
 Tabelle & mit & drei Spalten \\ \hline
 aber & nur mit zwei & Zeilen \\ \hline
\end{tabular}
\end{verbatim}
\end{codeblock}
\pause\bigskip

  \begin{table}
  \center


    \begin{tabular}{|l|c|r|}\hline
      Tabelle & mit & drei Spalten \\\hline
      aber & nur mit zwei & Zeilen \\\hline
    \end{tabular}
    \end{table}
\end{frame}

\begin{frame}[fragile]
\frametitle{Tabellen}
\framesubtitle{Feste Spaltenbreite}
Auch feste Breiten von Spalten können vorgegeben werden:
\begin{codeblock}
\begin{verbatim}
\begin{tabular}{|c|p{2cm}|p{4cm}}
 1. S. & 2. S. mit 2cm & 3. S. mit 4cm
\end{tabular}
\end{verbatim}
\end{codeblock}
\pause\bigskip
    \begin{table}
    \center

    \begin{tabular}{|c|p{2cm}|p{4cm}}
      1. S. & 2. S. mit 2cm & 3. S. mit 4cm
    \end{tabular}
    \end{table}\pause
    \medskip

    Spalten mit fester Breite sind immer im Blocksatz!
\end{frame}

\subsection{Erweiterte Tabellenfunktionen}

\begin{frame}[fragile]
\frametitle{Erweiterte Tabellenfunktionen}
\begin{itemize}
    \item Mehrfache \verb+|+ in Spaltendefinition für entsprechende Linien \pause
    \item \cmd{multicolumn}\marg{Anzahl Spalten}\marg{Format}\marg{Inhalt} für Zellen über mehrere Spalten\pause
    \item \cmd{vline} für senkrechte Linien in Zellen
\end{itemize}

\pause
\begin{table}
\center
  \begin{tabular}{|l||*{3}{c|}|r|}
   \hline
    A & 1 & 2 & 3 & Ein \vline Beispiel \\\hline
    B & 4 & 5 & 6 & Weitere Zeile \\\hline \hline
    C & \multicolumn{3}{c||}{7 8 9} & Ende \\\hline
  \end{tabular}
  \end{table}
\end{frame}


\subsection{Table-Umgebung}
\begin{frame}[fragile]
\frametitle{Table-Umgebung}
\begin{itemize}
    \item Die \umg{table}-Umgebung funktioniert wie die \umg{figure}-Umgebung\pause
    \item Eigentliche Tabelle mit \umg{tabular}-Umgebung in \umg{table}-Umgebung
        stecken\pause
    \item Tabelle zentrieren: \cmd{centering} vor der \umg{tabular}-Umgebung\pause
    \item \cmd{caption} und \cmd{label} wie gehabt
\end{itemize}

\end{frame}
